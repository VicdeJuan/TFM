%%%%%%%%%%%%%%%%%%%%%%%%%%%%%%%%%%%%%%%%%%%%%%%%%%%%%%%%%%%%%%%%%%%%%%%%%%%%%%%
%% Plantilla de memoria en LaTeX para la ETSIT - Universidad Rey Juan Carlos
%%
%% Por Gregorio Robles <grex arroba gsyc.urjc.es>
%%     Grupo de Sistemas y Comunicaciones
%%     Escuela Técnica Superior de Ingenieros de Telecomunicación
%%     Universidad Rey Juan Carlos
%% (muchas ideas tomadas de Internet, colegas del GSyC, antiguos alumnos...
%%  etc. Muchas gracias a todos)
%%
%% La última versión de esta plantilla está siempre disponible en:
%%     https://github.com/gregoriorobles/plantilla-memoria
%%
%% Para obtener PDF, ejecuta en la shell:
%%   make
%% (las imágenes deben ir en PNG o JPG)

%%%%%%%%%%%%%%%%%%%%%%%%%%%%%%%%%%%%%%%%%%%%%%%%%%%%%%%%%%%%%%%%%%%%%%%%%%%%%%%%

\documentclass[a4paper, 12pt]{book}
%\usepackage[T1]{fontenc}
%\usepackage[utf8]{inputenc}


%\usepackage[T1]{fontenc} 
%\usepackage[utf8]{inputenc}
%\PrerenderUnicode{ÁáÉéÍíÓóÚúÑñ} % Para que salgan las tildes y demás mierdas en el título.

\usepackage{exmath}


\RequirePackage{MathUnicode} % Paquete para poder poner caracteres griegos y demás cosas raras.
\usepackage[a4paper, left=3.8cm, right=2.5cm, top=3cm, bottom=3cm]{geometry}
\usepackage{times}

%\usepackage[utf8]{inputenc}
\usepackage[spanish]{babel} % Comenta esta línea si tu memoria es en inglés
\usepackage{url}
%\usepackage[dvipdfm]{graphicx}
\usepackage{graphicx}
\usepackage{float}  %% H para posicionar figuras
\usepackage[nottoc, notlot, notlof, notindex]{tocbibind} %% Opciones de índice
\usepackage{latexsym}  %% Logo LaTeX

\usepackage{amsmath} % Matemáticas
\usepackage{amsfonts} % Mas matemáticas
\usepackage{amssymb} % Mas matemáticas
\usepackage{amsthm} % Otro paquete de matemáticas
\usepackage[shadow]{todonotes} % Marcas To Do (a hacer)
\usepackage{xspace} % Espaciado correcto detras de comandos.
\usepackage{mathtools} % Arregla bastantes cosas de los entornos matemáticos por 
\usepackage[acronym]{glossaries} % Glosarios
\usepackage{makeidx}
\usepackage[breaklinks=true]{hyperref}
\usepackage{breakcites}

\interfootnotelinepenalty=10000


\newcommand{\nombreautor}{V\'ictor de Juan Sanz\xspace}
\newcommand{\nombretutor}{Desir\'e Garc\'ia L\'azaro\xspace}
\newcommand{\titulo}{Gamificaci\'on: visión general y aplicación en el aula de Matemáticas\xspace}
\newcommand{\TFM}{Trabajo de Fin de M\'aster\xspace}
\newcommand{\master}{M\'aster en Formaci\'on del Profesorado, especialidad Matem\'aticas\xspace}
%\newcommand{\master}{Máster en Formación del Profesorado: especialidad Matemáticas}

\renewcommand{\rmdefault}{ppl}
\renewcommand{\sfdefault}{fla}
\renewcommand{\ttdefault}{lmtt}
\renewcommand*{\familydefault}{\rmdefault}
\def\@fontsizeopt{11pt}
\message{Loading Palatino fonts}

\newcommand{\coment}[1]{\textit{#1}}

\renewcommand{\baselinestretch}{1.5}  %% Interlineado

\makeglossaries

\newacronym{PISA}{PISA}{\textit{Programme for International Student Assessment}}
\newacronym{TIMSS}{TIMSS}{\textit{Trends in International Mathematics and Science Study}}

\newacronym{INEE}{INEE}{Instituto Nacional de Evaluaci\'on Educativa}
\newacronym{INTEF}{INTEF}{Instituto Nacional de Tecnolog\'ias Educativas y Formaci\'on del Profesorado}


\newacronym{PDA}{PDA}{Programaci\'on Did\'actica Anual}


\newacronym{PBL}{PBL}{\textit{Points, Badges and Leaderboards} (Puntos, Medallas y Rankings)}

\newacronym{MUD}{MUD}{\textit{MultiUser Dungeon} -- juegos de rol online}

\newacronym{WCW}{WCW}{\textit{World Gamification Congress} -- Barcelona}

\newacronym{MOOC}{MOOC}{\textit{Massive Online Open Course} -- Curso masivo abierto en l\'inea}

\newacronym{TIC}{TIC}{Tecnolog\'ias de la Informaci\'on y de la Comunicaci\'on}



\newacronym{LOMCE}{LOMCE}{Ley Org\'anica 8/2013, de 9 de diciembre, para la Mejora de la Calidad Educativa}

\newacronym{BOE}{BOE}{Real Decreto 1105/2014, de 26 de diciembre, por el que se establece el curr\'iculo b\'asico de la Educaci\'on Secundaria Obligatoria y del Bachillerato}

\newacronym{BOCM}{BOCM}{Decreto 48/2015, de 14 de mayo, del Consejo de Gobierno, por el que se establece para la Comunidad de Madrid el curr\'iculo de la Educaci\'on Secundaria Obligatoria}

\newacronym[plural=EAEs,firstplural=Est\'andares de Aprendizaje Evaluables]{EAE}{EAE}{Est\'andar de Aprendizaje Evaluable}

\makeindex

\newif\iftocs
\tocstrue % Incluye en el índice general: la lista de tablas, lista de figuras y acrónimos.
\tocsfalse % Excluye en el índice general: la lista de tablas, lista de figuras y acrónimos.



\begin{document}
\pagenumbering{Roman} % para empezar la numeración de página con números
\renewcommand{\refname}{Bibliografía}  %% Renombrando
\renewcommand{\appendixname}{Apéndice}

%%%%%%%%%%%%%%%%%%%%%%%%%%%%%%%%%%%%%%%%%%%%%%%%%%%%%%%%%%%%%%%%%%%%%%%%%%%%%%%%
% PORTADA

\begin{titlepage}
\begin{center}
\begin{tabular}[c]{c c}
%\includegraphics[bb=0 0 194 352, scale=0.25]{logo} &
\includegraphics[scale=0.25]{img/logo_vect.png} &
\begin{tabular}[b]{l}
\Huge
\textsf{UNIVERSIDAD} \\
\Huge
\textsf{REY JUAN CARLOS} \\
\end{tabular}
\\
\end{tabular}

\vspace{3cm}

\master

\vspace{0.4cm}

\large
Curso Académico 2016/2017

\vspace{0.8cm}

\TFM

\vspace{2.5cm}

\LARGE
\textsc{\titulo}

\vspace{4cm}

\large
Autor : \nombreautor \\
Tutor : \nombretutor
\end{center}
\end{titlepage}

\newpage
\mbox{}
\thispagestyle{empty} % para que no se numere esta pagina


%%%%%%%%%%%%%%%%%%%%%%%%%%%%%%%%%%%%%%%%%%%%%%%%%%%%%%%%%%%%%%%%%%%%%%%%%%%%%%%%
%%%% Para firmar
\clearpage
%\pagenumbering{gobble}
\chapter*{}
\thispagestyle{empty} % para que no se numere esta pagina

\vspace{-4cm}
\begin{center}
\LARGE
\textbf{\TFM}

\vspace{1cm}
\large
\titulo

\vspace{1cm}
\large
\textbf{Autor :} \nombreautor \\
\textbf{Tutor :} \nombretutor

\end{center}

\vspace{1cm}
La defensa del presente \TFM se realizó el día \qquad$\;\,$ de \qquad\qquad\qquad\qquad \newline de 2017, siendo calificada por el siguiente tribunal:


\vspace{0.5cm}
\textbf{Presidente:}

\vspace{1.2cm}
\textbf{Secretario:}

\vspace{1.2cm}
\textbf{Vocal:}


\vspace{1.2cm}
y habiendo obtenido la siguiente calificación:

\vspace{1cm}
\textbf{Calificación:}


\vspace{1cm}
\begin{flushright}
Fuenlabrada, a \qquad$\;\,$ de \qquad\qquad\qquad\qquad de 2017
\end{flushright}

%%%%%%%%%%%%%%%%%%%%%%%%%%%%%%%%%%%%%%%%%%%%%%%%%%%%%%%%%%%%%%%%%%%%%%%%%%%%%%%%
%%%% Dedicatoria

\chapter*{}
%\pagenumbering{Roman} % para comenzar la numeracion de paginas en numeros romanos
\begin{flushright}
\textit{Dedicado a \\
mi familia / mi abuelo / mi abuela}
\end{flushright}

%%%%%%%%%%%%%%%%%%%%%%%%%%%%%%%%%%%%%%%%%%%%%%%%%%%%%%%%%%%%%%%%%%%%%%%%%%%%%%%%
%%%% Agradecimientos

\chapter*{Agradecimientos}
\input{tex/agradecimientos}


%%%%%%%%%%%%%%%%%%%%%%%%%%%%%%%%%%%%%%%%%%%%%%%%%%%%%%%%%%%%%%%%%%%%%%%%%%%%%%%%
%%%% Resumen

\chapter*{Resumen}
%!TEX root = ../TFM.tex
%\addcontentsline{toc}{chapter}{Resumen} % si queremos que aparezca en el ?dice
\markboth{RESUMEN}{RESUMEN} % encabezado


En el sistema educativo español se hace necesario mejorar su calidad e innovar en las metodologías utilizadas. 
%
Tanto es así que la última ley de educación, Ley Orgánica para la Mejora de la Calidad Educativa, lo tiene incluido en su propio nombre.

Aunque todavía no suficientes, cada vez son más los docentes y las instituciones que incorporan los nuevos avances de la investigación a sus prácticas docentes en las aulas.
%
Es importante apoyar esas iniciativas y dotarlas de recursos y posibilidades.
%
El presente trabajo trata precisamente eso, ofrecer un estudio detallado de la Gamificación como estrategia metodológica para el aula seguido de una propuesta concreta, diseñada y detallada lista para ser implementada directamente en el aula de Matemáticas Orientadas a las Ciencias Académicas de 3º de la ESO.


\begin{keywordsEs}
Gamificación, Taxonomía de Jugadores, Teorías de la Motivación y del Flow, Informes PISA, Unidad Didáctica, Matemáticas orientadas a las Ciencias Académicas, ESO, LOMCE
\end{keywordsEs}


%%%%%%%%%%%%%%%%%%%%%%%%%%%%%%%%%%%%%%%%%%%%%%%%%%%%%%%%%%%%%%%%%%%%%%%%%%%%%%%%
%%%% Resumen en inglés

\chapter*{Summary}
%\addcontentsline{toc}{chapter}{Resumen} % si queremos que aparezca en el ?dice
\markboth{RESUMEN}{RESUMEN} % encabezado

Here comes the abstract in English. 
%
The same text, it just need to be translated.
\cleardoublepage

%%%%%%%%%%%%%%%%%%%%%%%%%%%%%%%%%%%%%%%%%%%%%%%%%%%%%%%%%%%%%%%%%%%%%%%%%%%%%%%%
%%%%%%%%%%%%%%%%%%%%%%%%%%%%%%%%%%%%%%%%%%%%%%%%%%%%%%%%%%%%%%%%%%%%%%%%%%%%%%%%
% ÍNDICES %
%%%%%%%%%%%%%%%%%%%%%%%%%%%%%%%%%%%%%%%%%%%%%%%%%%%%%%%%%%%%%%%%%%%%%%%%%%%%%%%%

% Las buenas noticias es que los índices se generan automáticamente.
% Lo único que tienes que hacer es elegir cuáles quieren que se generen,
% y comentar/descomentar esa instrucción de LaTeX.
%%%% Índice de figuras


%%%% Índice de contenidos
\tableofcontents 


%%%% Índice de figuras
\cleardoublepage

\iftocs
\addcontentsline{toc}{chapter}{Lista de figuras.}% para que aparezca en el indice de contenidos
\else
\fi

\listoffigures % indice de figuras



%%%% Índice de tablas
\cleardoublepage

\iftocs
\addcontentsline{toc}{chapter}{Lista de tablas}% para que aparezca en el indice de contenidos
\else
\fi
\listoftables % indice de tablas

\iftocs
\addcontentsline{toc}{chapter}{Acrónimos}% para que aparezca en el indice de contenidos
\else
\fi

 
\printglossary[title=Glosario,toctitle=Glosario]
\printglossary[title=Acrónimos,toctitle=Acrónimos,type=\acronymtype]


%%%%%%%%%%%%%%%%%%%%%%%%%%%%%%%%%%%%%%%%%%%%%%%%%%%%%%%%%%%%%%%%%%%%%%%%%%%%%%%%
%%%%%%%%%%%%%%%%%%%%%%%%%%%%%%%%%%%%%%%%%%%%%%%%%%%%%%%%%%%%%%%%%%%%%%%%%%%%%%%%
% INTRODUCCIÓN %
%%%%%%%%%%%%%%%%%%%%%%%%%%%%%%%%%%%%%%%%%%%%%%%%%%%%%%%%%%%%%%%%%%%%%%%%%%%%%%%%

\cleardoublepage
\pagenumbering{arabic} % para empezar la numeración de página con números
\chapter{Introducción}
\label{chap:intro} % etiqueta para poder referenciar luego en el texto con ~\ref{sec:intro}


\section{Estado actual de la Educación de Matemáticas en España}
\label{sec:EstadoEducacionMates}

Todo va fatal y hay que hacer algo. 
%
Informes PISA \cite{InformePisa} Y TIMSS \cite{InformeTimss}.
%
Además, las actitudes hacia las matemáticas no suelen ser buenas. Hay miedos, indefensión ... ¿Depende del profesor? ¿Depende del alumno? En \cite{ActitudesHaciaMates} se contestan algunas de estas preguntas y más.

Este trabajo se propone ofrecer una propuesta de intervención en el aula concreta y detallada con la que intentar atajar eso.

Para ello, procedemos a describir el marco teórico en el que se enmarca la metodología.


\section{La Gamificación}



\coment{Esto es lo que correspondería al marco teórico.}

En esta sección procedemos a describir a nivel general la Gamificación. 
%
La Gamificación entendida como una técnica que puede ser aplicada a muchos ámbitos, no sólo al educativo. 
%
Primeramente trataremos de clarificar el concepto de gamificación, porqué y cómo se puede gamificar un contexto.
%
Una vez detallado, en la siguiente sección trataremos de aplicar la gamificación al aula.



\coment{Qué es la Gamificación}

Gamificación o ludificación, optamos por Gamificación porque...

Atendiendo a \cite{GamificationDef} podemos definir \concept{Gamificación} como \textit{el uso de elementos de juegos en contextos no lúdicos}. 

Algunos de los elementos que hacen atractivos los juegos son
\textbf{Feedback instantáneo}
poder tomar \textbf{decisiones importantes},
\textbf{bucles de atracción y de fracaso},
\textbf{niveles de dificultad progresiva}... La idea es incorporar esos elementos en algún contexto no lúdico (en general, no sólo en la educación\footnote{No querría centrarme sólo en la educación para dar una visión más general de la gamificación como técnica en muchos campos: marketing, negocios... Creo que ofrece un punto de vista más riguroso y científico hablar de Gamificación en general y no sólo Gamificación en la educación.}).


\coment{Por qué gamificar}

Alguna lectura sobre porqué gamificar conceptualmente.

Porque funciona. Conclusiones de \cite{EmpiricalGamification}


\subsection{Diferencias entre Juegos serios y Gamificación}

Son conceptos distintos. Los \concept[Juegos serios]{juegos serios} son \cite{GamificationDef}... mientras que la gamificación consiste en...

Aunque los juegos serios pueden ser interesantes (conclusiones de \cite{MetaSerious}), se salen de este estudio.


\subsection{Elementos de la Gamificación}

\paragraph{Pensando como un diseñador}
\paragraph{La anatomía de la diversión}

Los elementos de la Gamificación son los elementos que podemos encontrar en los juegos, atendiendo a \cite{Hunicke04mda:a} son dinámicas, mecánicas y componentes.


\subsubsection{Dinámicas}

\subsubsection{Mecánicas}

\subsubsection{Componentes}



\paragraph{Puntos ...} \gls{pbl}

\subparagraph*{Puntos}

\subparagraph*{Medallas}

\subparagraph*{Leaderboards}

Los leaderboards desmotivan al 80\% de la gente. 

Podríamos definir competición utilizando la definición de \cite{Crawford_CompetitionDef} Competition is when students are “constrained from impeding each other and instead devote the entirety of their attentions to optimizing their own performance.”

Crean una competición con la que hay que tener cuidado. Leyendo \cite{CompetitionInEd}...

\subsubsection{Impacto de la motivación}

\paragraph{Posibles peligros}

Ya hemos constatado un posible peligro al describir el componente \textit{leaderboard}. Además, hay otros.

Más competidores, menos motivación. \cite{n-effect}

Premios para los ganadores deberían ser de poca importancia o incluso simbólicos para asegurar que el esfuerzo de los estudiantes es intrínseco y no está dirigido por la expetativa del premio \cite{CompetitionInEd}.

\section{Educación gamificada}

Hasta ahora hemos hablado en términos generales de la gamificación. 
%
En esta sección trataremos de estudiar cómo se adapta esa teoría general al ecosistema de un aula.

\cite{lee2011gamification} constata varias cosas.

Hay otros estudios interesantes como \cite{Hanus2015152} y \cite{ReviewGamificationInEducation}.

\subsection{Potencial de la gamificación en el aula}

En la sección \ref{sec:EstadoEducacionMates} hemos constatado algunos problemas. 
%
Vamos a ir viendo cómo la Gamificación puede atajar esos problemas concretos y también, qué problemas se quedarían sin atajar.

\subsubsection{¿Es una metodología inclusiva?}

Esta pregunta es importante, pero no hay que descartar la metodología en caso de que la respuesta sea negativa, ya que también es importante plantearse:  ¿Es una metodología \textbf{más inclusiva} que la tradicional?
%
Tal vez no es una metodología perfecta, pero sí una metodología que aporta mejoras.



%%%%%%%%%%%%%%%%%%%%%%%%%%%%%%%%%%%%%%%%%%%%%%%%%%%%%%%%%%%%%%%%%%%%%%%%%%%%%%%%
%%%%%%%%%%%%%%%%%%%%%%%%%%%%%%%%%%%%%%%%%%%%%%%%%%%%%%%%%%%%%%%%%%%%%%%%%%%%%%%%
% DISEÑO E IMPLEMENTACIÓN %
%%%%%%%%%%%%%%%%%%%%%%%%%%%%%%%%%%%%%%%%%%%%%%%%%%%%%%%%%%%%%%%%%%%%%%%%%%%%%%%%

\cleardoublepage
\chapter{Gamificar las Matemáticas}

\section{¿Se pueden gamificar las Matemáticas?}

Lo siguiente que necesitamos, si estamos convencidos de que la Gamificación puede ser una buena metodología, es tratar cómo se puede gamificar un aula.
%
Dedicaremos el próximo capítulo a ello.

\section{Estado de la Ley y del currículo}

Primero: ¿Permite o incluso anima la ley al empleo de esta metodología?

Segundo: Necesitamos elegir un curso y su currículo.

\subsection{Elección del curso para Gamificar}

El curso elegido es: porque...

\subsubsection{Contenidos de ese curso}
\subsubsection{Estándares de Aprendizaje Evaluables}


\section{Elementos de la gamificación aplicables al aula de Matemáticas}

\paragraph{Dinámicas}

\subparagraph{Narrativa}
\subparagraph{Progresión}

\paragraph{Mecánicas}

Las mecánicas que utilizaremos serán:
\subparagraph{Retos},
\subparagraph{Competición y cooperación},
\subparagraph{Feedback},
\subparagraph{Premios}

 Los premios serán: logros, medallas, puntos (que describiremos a continuación).

\paragraph{Elementos}

Los elementos que utilizaremos serán:
\subparagraph{Niveles},
\subparagraph{Logros},
\subparagraph{Puntos},
\subparagraph{Medallas},
\subparagraph{Jefes finales}...

\section{Programación anual Gamificada}

\subsection{Aprendizaje competencial}
\subsection{Cobertura de los contenidos}
\subsection{Incorporación de los Estándares de Aprendizaje Evaluables}

Matriz de trazabilidad entre los Estándares de Aprendizaje Evaluables y los componentes de la Gamificación:
\begin{table}[hbtp]
\begin{tabular}{c|c}
& Estándares de Aprendizaje Evaluables\\\hline
Elementos & \\\hline
\end{tabular}
\label{Trazabilidad}
\caption{Matriz de trazabilidad entre los Estándares de Aprendizaje Evaluables y los componentes de la Gamificación.}
\end{table}


\section{Evaluación de la gamificación}

Es fundamental saber si la gamificación tal como se ha llevado a cabo ha merecido la pena, qué aspectos son mejorables y cuáles hay que mantener.
%
Podríamos utilizar el marco propuesto por \cite{EvaluacionGamificacion} para autoevaluar la gamificación realizada.


%%%%%%%%%%%%%%%%%%%%%%%%%%%%%%%%%%%%%%%%%%%%%%%%%%%%%%%%%%%%%%%%%%%%%%%%%%%%%%%%
%%%%%%%%%%%%%%%%%%%%%%%%%%%%%%%%%%%%%%%%%%%%%%%%%%%%%%%%%%%%%%%%%%%%%%%%%%%%%%%%
% CONCLUSIONES %
%%%%%%%%%%%%%%%%%%%%%%%%%%%%%%%%%%%%%%%%%%%%%%%%%%%%%%%%%%%%%%%%%%%%%%%%%%%%%%%%

\cleardoublepage
\chapter{Conclusiones}
\label{chap:conclusiones}


%Se comenta pero no se pone como apartado específico: 
% \section{Trabajos futuros}

\label{sec:trabajos_futuros}

Puesta en marcha y evaluación. Detectar posibles mejoras.

Propuestas de Gamificación para otros cursos y otras asignaturas.



%%%%%%%%%%%%%%%%%%%%%%%%%%%%%%%%%%%%%%%%%%%%%%%%%%%%%%%%%%%%%%%%%%%%%%%%%%%%%%%%
%%%%%%%%%%%%%%%%%%%%%%%%%%%%%%%%%%%%%%%%%%%%%%%%%%%%%%%%%%%%%%%%%%%%%%%%%%%%%%%%
% APÉNDICE(S) %
%%%%%%%%%%%%%%%%%%%%%%%%%%%%%%%%%%%%%%%%%%%%%%%%%%%%%%%%%%%%%%%%%%%%%%%%%%%%%%%%

\cleardoublepage
\appendix


%%%%%%%%%%%%%%%%%%%%%%%%%%%%%%%%%%%%%%%%%%%%%%%%%%%%%%%%%%%%%%%%%%%%%%%%%%%%%%%%
%%%%%%%%%%%%%%%%%%%%%%%%%%%%%%%%%%%%%%%%%%%%%%%%%%%%%%%%%%%%%%%%%%%%%%%%%%%%%%%%
% BIBLIOGRAFIA %
%%%%%%%%%%%%%%%%%%%%%%%%%%%%%%%%%%%%%%%%%%%%%%%%%%%%%%%%%%%%%%%%%%%%%%%%%%%%%%%%

\cleardoublepage

% Las siguientes dos instrucciones es todo lo que necesitas
% para incluir las citas en la memoria
\bibliographystyle{apalike}
\bibliography{memoria}  % memoria.bib es el nombre del fichero que contiene
% las referencias bibliográficas. Abre ese fichero y mira el formato que tiene,
% que se conoce como BibTeX. Hay muchos sitios que exportan referencias en
% formato BibTeX. Prueba a buscar en http://scholar.google.com por referencias
% y verás que lo puedes hacer de manera sencilla.
% Más información: 
% http://texblog.org/2014/04/22/using-google-scholar-to-download-bibtex-citations/

\printindex

\end{document}
