%%%%%%%%%%%%%%%%%%%%%%%%%%%%%%%%%%%%%%%%%%%%%%%%%%%%%%%%%%%%%%%%%%%%%%%%%%%%%%%
%% Plantilla de memoria en LaTeX para la ETSIT - Universidad Rey Juan Carlos
%%
%% Por Gregorio Robles <grex arroba gsyc.urjc.es>
%%     Grupo de Sistemas y Comunicaciones
%%     Escuela Técnica Superior de Ingenieros de Telecomunicación
%%     Universidad Rey Juan Carlos
%% (muchas ideas tomadas de Internet, colegas del GSyC, antiguos alumnos...
%%  etc. Muchas gracias a todos)
%%
%% La última versión de esta plantilla está siempre disponible en:
%%     https://github.com/gregoriorobles/plantilla-memoria
%%
%% Para obtener PDF, ejecuta en la shell:
%%   make
%% (las imágenes deben ir en PNG o JPG)

%%%%%%%%%%%%%%%%%%%%%%%%%%%%%%%%%%%%%%%%%%%%%%%%%%%%%%%%%%%%%%%%%%%%%%%%%%%%%%%%

\documentclass[a4paper, 12pt]{book}
%\usepackage[T1]{fontenc}
%\usepackage[utf8]{inputenc}


%\usepackage[T1]{fontenc} 
%\usepackage[utf8]{inputenc}
%\PrerenderUnicode{ÁáÉéÍíÓóÚúÑñ} % Para que salgan las tildes y demás mierdas en el título.

\usepackage{exmath}


\RequirePackage{MathUnicode} % Paquete para poder poner caracteres griegos y demás cosas raras.
%PRETTY:\usepackage[a4paper, left=3.8cm, right=2.5cm, top=3.3cm, bottom=3cm,headsep=1cm,headheight=3cm]{geometry}
%%% UGLY
\usepackage[a4paper, left=2.5cm, right=2.5cm, top=3.3cm, bottom=3cm,headsep=1cm,headheight=3cm]{geometry}
\usepackage{times}

%\usepackage[utf8]{inputenc}
\usepackage[spanish]{babel} % Comenta esta línea si tu memoria es en inglés
\usepackage{url}
%\usepackage[dvipdfm]{graphicx}
\usepackage{graphicx}
\usepackage{float}  %% H para posicionar figuras
\usepackage[nottoc, notlot, notlof, notindex]{tocbibind} %% Opciones de índice
\usepackage{latexsym}  %% Logo LaTeX

\usepackage{amsmath} % Matemáticas
\usepackage{amsfonts} % Mas Matemáticas
\usepackage{amssymb} % Mas Matemáticas
\usepackage{amsthm} % Otro paquete de Matemáticas
\usepackage[shadow]{todonotes} % Marcas To Do (a hacer)
\usepackage{xspace} % Espaciado correcto detras de comandos.
\usepackage{mathtools} % Arregla bastantes cosas de los entornos matemáticos por 
\usepackage[acronym]{glossaries} % Glosarios
\usepackage{makeidx}
\usepackage[breaklinks=true,
			pdfauthor={V\'ictor de Juan},
            pdftitle={Trabajo de Fin de M\'aster},
            pdfsubject={Gamificaci\'on},
            pdfkeywords={Gamificaci\'on, Educaci\'on, Matem\'aticas}]{hyperref}
			
\usepackage{breakcites}




\interfootnotelinepenalty=10000

\usepackage{placeins}
\usepackage{flafter}

\usepackage{fancyhdr}
 
\newcommand{\nombreautor}{V\'ictor de Juan Sanz\xspace}
\newcommand{\nombretutor}{Desir\'e Garc\'ia L\'azaro\xspace}
\newcommand{\titulo}{Gamificaci\'on: visión general y aplicación en el aula de Matemáticas\xspace}
\newcommand{\TFM}{Trabajo de Fin de M\'aster\xspace}
\newcommand{\master}{M\'aster en Formaci\'on del Profesorado, especialidad Matem\'aticas\xspace}
%\newcommand{\master}{Máster en Formación del Profesorado: especialidad Matemáticas}

\usepackage{newtxmath,newtxtext}
\usepackage{seqsplit}

\pagestyle{fancy}
\fancyhf{}
\rhead{\raisebox{1.15\height}{\nombreautor\quad\quad}}
\lhead{\raisebox{-0.13\height}{\quad\quad\includegraphics[height=15mm]{img/Logo_header.png}}}
\chead{Gamificaci\'on: visión general y\\
aplicación en el aula de Matemáticas}
\cfoot{\thepage}
\renewcommand{\headrulewidth}{0.4pt}
\renewcommand{\headrule}{{
\vspace{0.3cm}\hrule width\headwidth height\headrulewidth \vskip-\headrulewidth}}


\newcommand{\coment}[1]{\textit{#1}}
\renewcommand{\coment}[1]{}
%\newcommand{\citePisa}[1]{\cite{InformePisa#1}}
\newcommand{\citePisa}[1]{(\gls{PISA}, #1)}


\renewcommand{\baselinestretch}{1.5}  %% Interlineado

\newcommand{\ign}[1]{}

\makeglossaries

\newacronym{PISA}{PISA}{\textit{Programme for International Student Assessment}}
\newacronym{TIMSS}{TIMSS}{\textit{Trends in International Mathematics and Science Study}}

\newacronym{INEE}{INEE}{Instituto Nacional de Evaluaci\'on Educativa}
\newacronym{INTEF}{INTEF}{Instituto Nacional de Tecnolog\'ias Educativas y Formaci\'on del Profesorado}


\newacronym{PDA}{PDA}{Programaci\'on Did\'actica Anual}


\newacronym{PBL}{PBL}{\textit{Points, Badges and Leaderboards} (Puntos, Medallas y Rankings)}

\newacronym{MUD}{MUD}{\textit{MultiUser Dungeon} -- juegos de rol online}

\newacronym{WCW}{WCW}{\textit{World Gamification Congress} -- Barcelona}

\newacronym{MOOC}{MOOC}{\textit{Massive Online Open Course} -- Curso masivo abierto en l\'inea}

\newacronym{TIC}{TIC}{Tecnolog\'ias de la Informaci\'on y de la Comunicaci\'on}



\newacronym{LOMCE}{LOMCE}{Ley Org\'anica 8/2013, de 9 de diciembre, para la Mejora de la Calidad Educativa}

\newacronym{BOE}{BOE}{Real Decreto 1105/2014, de 26 de diciembre, por el que se establece el curr\'iculo b\'asico de la Educaci\'on Secundaria Obligatoria y del Bachillerato}

\newacronym{BOCM}{BOCM}{Decreto 48/2015, de 14 de mayo, del Consejo de Gobierno, por el que se establece para la Comunidad de Madrid el curr\'iculo de la Educaci\'on Secundaria Obligatoria}

\newacronym[plural=EAEs,firstplural=Est\'andares de Aprendizaje Evaluables]{EAE}{EAE}{Est\'andar de Aprendizaje Evaluable}

\makeindex

\newif\iftocs
\tocstrue % Incluye en el índice general: la lista de tablas, lista de figuras y acrónimos.
\tocsfalse % Excluye en el índice general: la lista de tablas, lista de figuras y acrónimos.



\begin{document}

%%%%%%%%%%%%%%%%%%%%%%%%%%%%%%%%%%%%%%%%%%%%%%%%%%%%%%%%%%%%%%%%%%%%%%%%%%%%%%%%
%%%%%%%%%%%%%%%%%%%%%%%%%%%%%%%%%%%%%%%%%%%%%%%%%%%%%%%%%%%%%%%%%%%%%%%%%%%%%%%%
% PORTADAS Y DEMÁS %
%%%%%%%%%%%%%%%%%%%%%%%%%%%%%%%%%%%%%%%%%%%%%%%%%%%%%%%%%%%%%%%%%%%%%%%%%%%%%%%%


\pagenumbering{Roman} % para empezar la numeración de página con números
\renewcommand{\refname}{Bibliografía}  %% Renombrando
\renewcommand{\appendixname}{Apéndice}

%%%%%%%%%%%%%%%%%%%%%%%%%%%%%%%%%%%%%%%%%%%%%%%%%%%%%%%%%%%%%%%%%%%%%%%%%%%%%%%%
% PORTADA

\begin{titlepage}
\begin{center}
\Huge
\textsc{\textbf{Universidad Rey Juan Carlos}} \\
\vspace{1cm}
\includegraphics[scale=1]{img/logoportada.png}

\vspace{1cm}
\large\textsc{\textbf{\TFM}}

\vspace{0.7cm}

\begin{mdframed}
\begin{center}
\vspace{0.5cm}
\textsc{\titulo}
\vspace{0.5cm}
\end{center}
\end{mdframed}

\textsc{\master}

\large
\especialidad

Curso Académico 2016/2017

\end{center}
\begin{tabular}{p{4cm}p{10cm}}
 &  Apellidos y Nombre del Alumno:\\

 &  \vspace{-0.8cm}\begin{mdframed}\quad\quad de Juan Sanz, Víctor\hfill\end{mdframed}\\
 &  \vspace{-0.6cm}Directora TFM:\\
 &  \vspace{-1.1cm}\begin{mdframed}\quad\quad García Lázaro, Desiré\hfill\end{mdframed}
\end{tabular}


\end{titlepage}



%%%%%%%%%%%%%%%%%%%%%%%%%%%%%%%%%%%%%%%%%%%%%%%%%%%%%%%%%%%%%%%%%%%%%%%%%%%%%%%%
%%%% Para firmar

%\pagenumbering{gobble}
\newpage
\mbox{}
\thispagestyle{empty} % para que no se numere esta pagina

\ign{
\clearpage
\chapter*{}
\thispagestyle{empty} % para que no se numere esta pagina

\vspace{-4cm}
\begin{center}
\LARGE
\textbf{\TFM}

\vspace{1cm}
\large
\titulo

\vspace{1cm}
\large
\textbf{Autor :} \nombreautor \\
\textbf{Tutor :} \nombretutor

\end{center}

%\vspace{1cm}
%La defensa del presente \TFM se realizó el día \qquad$\;\,$ de \qquad\qquad\qquad\qquad \newline de 2017, siendo calificada por el siguiente tribunal:
%\vspace{0.5cm}
%\textbf{Presidente:}
%\vspace{1.2cm}
%\textbf{Secretario:}
%\vspace{1.2cm}
%\textbf{Vocal:}
%\vspace{1.2cm} y habiendo obtenido la siguiente calificación:
%\vspace{1cm}
%\textbf{Calificación:}
%\vspace{1cm}
%\begin{flushright} Fuenlabrada, a \qquad$\;\,$ de \qquad\qquad\qquad\qquad de 2017
%\end{flushright}

%%%%%%%%%%%%%%%%%%%%%%%%%%%%%%%%%%%%%%%%%%%%%%%%%%%%%%%%%%%%%%%%%%%%%%%%%%%%%%%%
%%%% Dedicatoria

\chapter*{}
%\pagenumbering{Roman} % para comenzar la numeracion de paginas en numeros romanos
\begin{flushright}
\textit{Dedicado a \\
mi familia / mi abuelo / mi abuela}
\end{flushright}
}
%%%%%%%%%%%%%%%%%%%%%%%%%%%%%%%%%%%%%%%%%%%%%%%%%%%%%%%%%%%%%%%%%%%%%%%%%%%%%%%%
%%%% Agradecimientos
\pagestyle{empty}
\ign{\chapter*{Agradecimientos}\input{tex/agradecimientos}}



%%%%%%%%%%%%%%%%%%%%%%%%%%%%%%%%%%%%%%%%%%%%%%%%%%%%%%%%%%%%%%%%%%%%%%%%%%%%%%%%
%%%%%%%%%%%%%%%%%%%%%%%%%%%%%%%%%%%%%%%%%%%%%%%%%%%%%%%%%%%%%%%%%%%%%%%%%%%%%%%%
% ÍNDICES %
%%%%%%%%%%%%%%%%%%%%%%%%%%%%%%%%%%%%%%%%%%%%%%%%%%%%%%%%%%%%%%%%%%%%%%%%%%%%%%%%

% Las buenas noticias es que los índices se generan automáticamente.
% Lo único que tienes que hacer es elegir cuáles quieren que se generen,
% y comentar/descomentar esa instrucción de LaTeX.
%%%% Índice de figuras


%%%% Índice de contenidos
\cleardoublepage
\renewcommand{\contentsname}{{\normalfont\fontsize{12}{15}\bfseries}Índice de Contenidos}

\fancypagestyle{plain}{}
\begingroup
  \cleardoublepage
  \pagestyle{fancy}
  \tableofcontents
  \cleardoublepage
\endgroup

%%%%%%%%%%%%%%%%%%%%%%%%%%%%%%%%%%%%%%%%%%%%%%%%%%%%%%%%%%%%%%%%%%%%%%%%%%%%%%%%
%%%% Resumen

\cleardoublepage
\chapter*{Resumen} %!TEX root = ../TFM.tex
%\addcontentsline{toc}{chapter}{Resumen} % si queremos que aparezca en el ?dice
\markboth{RESUMEN}{RESUMEN} % encabezado


En el sistema educativo español se hace necesario mejorar su calidad e innovar en las metodologías utilizadas. 
%
Tanto es así que la última ley de educación, Ley Orgánica para la Mejora de la Calidad Educativa, lo tiene incluido en su propio nombre.

Aunque todavía no suficientes, cada vez son más los docentes y las instituciones que incorporan los nuevos avances de la investigación a sus prácticas docentes en las aulas.
%
Es importante apoyar esas iniciativas y dotarlas de recursos y posibilidades.
%
El presente trabajo trata precisamente eso, ofrecer un estudio detallado de la Gamificación como estrategia metodológica para el aula seguido de una propuesta concreta, diseñada y detallada lista para ser implementada directamente en el aula de Matemáticas Orientadas a las Ciencias Académicas de 3º de la ESO.


\begin{keywordsEs}
Gamificación, Taxonomía de Jugadores, Teorías de la Motivación y del Flow, Informes PISA, Unidad Didáctica, Matemáticas orientadas a las Ciencias Académicas, ESO, LOMCE
\end{keywordsEs}


%%%%%%%%%%%%%%%%%%%%%%%%%%%%%%%%%%%%%%%%%%%%%%%%%%%%%%%%%%%%%%%%%%%%%%%%%%%%%%%%
%%%% Resumen en inglés

\chapter*{Summary} %\addcontentsline{toc}{chapter}{Resumen} % si queremos que aparezca en el ?dice
\markboth{RESUMEN}{RESUMEN} % encabezado

Here comes the abstract in English. 
%
The same text, it just need to be translated. 
\cleardoublepage





%%%% Índice de figuras
\cleardoublepage

\iftocs
\addcontentsline{toc}{chapter}{Lista de figuras.}% para que aparezca en el indice de contenidos
\else
\fi

\listoffigures % indice de figuras



%%%% Índice de tablas
\cleardoublepage

\iftocs
\addcontentsline{toc}{chapter}{Lista de tablas}% para que aparezca en el indice de contenidos
\else
\fi
\listoftables % indice de tablas

%\fancypagestyle{plain}{\cfoot{\thepage}}
%\iftocs
%\addcontentsline{toc}{chapter}{Acrónimos}% para que aparezca en el indice de contenidos
%\else
%\fi

 
\ifincgls
\printglossary[title=Glosario,toctitle=Glosario]
\printglossary[title=Acrónimos,toctitle=Acrónimos,type=\acronymtype]
\fi

%%%%%%%%%%%%%%%%%%%%%%%%%%%%%%%%%%%%%%%%%%%%%%%%%%%%%%%%%%%%%%%%%%%%%%%%%%%%%%%%
%%%%%%%%%%%%%%%%%%%%%%%%%%%%%%%%%%%%%%%%%%%%%%%%%%%%%%%%%%%%%%%%%%%%%%%%%%%%%%%%
% INTRODUCCIÓN %
%%%%%%%%%%%%%%%%%%%%%%%%%%%%%%%%%%%%%%%%%%%%%%%%%%%%%%%%%%%%%%%%%%%%%%%%%%%%%%%%

\cleardoublepage
\pagenumbering{arabic} % para empezar la numeración de página con números
\chapter{Introducción}
\label{sec:intro} % etiqueta para poder referenciar luego en el texto con ~\ref{sec:intro}
\pagenumbering{arabic} % para empezar la numeración de página con números

En este capítulo se introduce el proyeto. 
Debería tener información general sobre  el mismo, dando la información sobre el contexto en el que se ha desarrollado.

%No te olvides de echarle un ojo a la página con los cinco errores de escritura más frecuentes\footnote{\url{http://www.tallerdeescritores.com/errores-de-escritura-frecuentes}}.

\section{Ideas interesantes}

\begin{itemize}
	\item \url{https://openbadges.org/developers/} para clase de informática. 1 punto por badget conseguido. Hay 15 y tu eliges cuales quieres.
\end{itemize}

\section{Estructura de la memoria}
\label{sec:estructura}

En esta sección se debería introducir la esctura de la memoria. Así:

\begin{itemize}
  \item En el primer capítulo se hace una intro al proyecto.
  
  \item En el capítulo~\ref{chap:objetivos} se muestran los objetivos del proyecto.
  
  \item A continuación se presenta el estado del arte.
  
  \item \ldots
\end{itemize}








\section{La Gamificación}

\coment{Esto es lo que correspondería al marco teórico.}


En esta sección procedemos a describir en términos generales la Gamificación como estrategia metodológica.
%
Primeramente trataremos de clarificar el concepto de gamificación, porqué y cómo se puede gamificar un contexto.
%
Una vez detallado, en la siguiente sección trataremos de aplicar la gamificación en la educación.


La palabra \textit{Gamificación} es una traducción del término inglés \textit{Gamification}, palabra derivada del sustantivo \textit{game}.
%
En castellano no podemos considerar Gamificación como palabra derivada de otro sustantivo. 
%
Otra posible traducción sería ludificación, palabra derivada del adjetivo lúdico.
%
En este sentido podríamos utilizar la denominación ludificación, pero se ha preferido utilizar en esta tesis el término Gamificación para buscar la congruencia con la tendencia entre el profesorado.


\coment{Qué es la Gamificación}


Atendiendo a \cite{GamificationDef} podemos definir \concept{Gamificación} como \textit{el uso de elementos de juegos en contextos no lúdicos}. 

Esta definición no está restringida al ámbito educativo. 
%
De hecho, la Gamificación entendida como una estrategia o metodología es aplicada a día de hoy en diversos ámbitos. 
%
Por ejemplo,es una técnica muy utilizada en el campo del Marketing y de los Recursos Humanos.
%
Tanto la educación como el Marketing son contextos no lúdicos en los que al introducir de elementos de juegos estaríamos gamificando. 
%
Pero, ¿qué son elementos de juegos? 
%
Aquellas dinámicas, mecánicas y componentes que hacen atractivos los juegos.
%
Es importante clarificar que el término juegos engloba tanto juegos de mesa como videojuegos y que no todos los juegos tienen los mismos elementos.
%
Algunos juegos, por sus características, utilizan más unos determinados elementos que otros.
%
Con fines didácticos vamos a describir algunos de esos elementos, aunque más adelante se ofrecerá una lista más completa.

Por ejemplo, en los juegos se produce un feedback instantáneo: el jugador avanza un número de casillas, obtiene dinero al pasar por una casilla determinada, aparecen carteles con información sobre el desempeño (¡Sigue así!), etc.
%
Este feedback instantáneo no se produce en contextos no lúdicos.
%
Un trabajador de una empresa es evaluado (como mucho) una vez al año. 
%
¿Y si cada día recibiera un pequeño feedback sobre su rendimiento?


Otro elemento importante de algunos juegos es la posibilidad de tomar decisiones importantes.
%
De hecho, algunos juegos van modificando la historia del juego a medida que el jugador avanza. 
%
Esto ocurre en videojuegos (la saga Mass Effect sería un ejemplo) pero también en los tradicionales juegos de rol (Dragones y Mazmorras).
%
¿Trabajarán mejor los empleados de una empresa si dicha empresa si pueden elegir en qué trabajar?
%
Algunos grandes proyectos de Google han surgido como resultado del porcentaje de tiempo que la empresa establece para que sus trabajadores trabajen en sus propios proyectos.

Un consumidor puede elegir entre 2 fruterías con la misma variedad de frutas y precios similares.
%
Si en una de ellas cada semana hay una nueva oferta para un tipo de fruta específico y es la fruta de la semana podría despertar atracción en el consumidor, incluso, que el consumidor quiera que empiece una nueva semana para ver cuál es la nueva oferta de la frutería.
%
A este fenómeno se le denomina \concept{bucle de atracción}. 
%
Es muy típico en los videojuegos online que existan misiones diarias, misiones semanales e incluso eventos especiales en fechas especiales como podría ser Navidad o Primavera.
%
¿Mejoraría la motivación de los estudiantes de una asignatura que el profesor planteara un reto opcional cada semana?


\coment{Por qué gamificar}

\paragraph{¿Por qué gamificar?} Se han dejado algunas preguntas en el aire, pero todas ellas se pueden englobar en la siguiente pregunta: ¿La Gamificación en un determinado contexto concreto mejora el rendimiento de las personas en ese contexto?
%
Hay bastantes investigaciones recientes que intentan contestar a esta pregunta.
%
Recurriendo a una revisión bibliográfica \cite{EmpiricalGamification} encontramos que la mayoría de experimentos empíricos sobre Gamificación han tenido efectos positivos en términos motivacionales.
%
Sin embargo, no todos los estudios encontraron efectos positivos en todos los participantes.
%
Además, parece que la gamificación falla a largo plazo, tal vez por el efecto de la novedad. 
%
\footnote{Estos posibles peligros y otros se tratarán más adecuadamente en \ref{PosiblesPeligros}, cuando el lector disponga de visión más global de la gamificación.}

Pero la gamificación puede producir efectos beneficiosos más allá de la motivación.
% 
De acuerdo con el Profesor Kevin Werbach de la Universidad de Pensilvania la Gamificación permite fidelizar a las persona, hacer todavía más social el contexto, ofrecer a la persona un sentido del progreso en ese contexto y crear un hábito.


\subsection{Diferencias entre Juegos serios y Gamificación}

Es importante distinguir gamificación de juegos serios. Los \concept[Juegos serios]{juegos serios} consisten en la modificación del contexto transformándolo en un contexto lúdico, mientras que la gamificación incorpora elementos en un contexto no lúdico, manteniendo el contexto como no lúdico.
%
Un ejemplo de juego serio sería idear un juego de conquistas como el Risk para trabajar los mapas políticos con los alumnos. 

Aunque los juegos serios tengan consecuencias positivas y puedan ser una buena herramienta\footnote{Tanto es así que \cite{MetaSerious} concluye en su revisión que los juegos serios son más efectivos en contextos de aprendizaje pero menos efectivo que los métodos convencionales en términos motivacionales.}, su estudio se sale de esta tesis.

\subsection{Elementos de la Gamificación}

\paragraph{Pensando como un diseñador}

\paragraph{La anatomía de la diversión}

Los elementos de la Gamificación son los elementos que podemos encontrar en los juegos, atendiendo a \cite{Hunicke04mda:a} son dinámicas, mecánicas y componentes.

\subsubsection{Dinámicas}

\subsubsection{Mecánicas}

\subsubsection{Componentes}



\paragraph{Puntos ...} \gls{pbl}

\subparagraph*{Puntos}

\subparagraph*{Medallas}

\subparagraph*{Leaderboards}

Los leaderboards desmotivan al 80\% de la gente. 

Podríamos definir competición utilizando la definición de \cite{Crawford_CompetitionDef} Competition is when students are “constrained from impeding each other and instead devote the entirety of their attentions to optimizing their own performance.”

Crean una competición con la que hay que tener cuidado. Leyendo \cite{CompetitionInEd}...

\subsubsection{Impacto de la motivación}

\paragraph{Posibles peligros}
\label{PosiblesPeligros}

Ya hemos constatado un posible peligro al describir el componente \textit{leaderboard}. Además, hay otros.

Más competidores, menos motivación. \cite{n-effect}

Premios para los ganadores deberían ser de poca importancia o incluso simbólicos para asegurar que el esfuerzo de los estudiantes es intrínseco y no está dirigido por la expetativa del premio \cite{CompetitionInEd}.

\section{Educación gamificada}

Hasta ahora hemos hablado en términos generales de la gamificación. 
%
En esta sección trataremos de estudiar cómo se adapta esa teoría general al ecosistema de un aula.

\cite{lee2011gamification} constata varias cosas.

Hay otros estudios interesantes como \cite{Hanus2015152} y \cite{ReviewGamificationInEducation}.

\subsection{Potencial de la gamificación en el aula}

En la sección \ref{sec:EstadoEducacionMates} hemos constatado algunos problemas. 
%
Vamos a ir viendo cómo la Gamificación puede atajar esos problemas concretos y también, qué problemas se quedarían sin atajar.

\subsubsection{¿Es una metodología inclusiva?}

Esta pregunta es importante, pero no hay que descartar la metodología en caso de que la respuesta sea negativa, ya que también es importante plantearse:  ¿Es una metodología \textbf{más inclusiva} que la tradicional?
%
Tal vez no es una metodología perfecta, pero sí una metodología que aporta mejoras.



%%%%%%%%%%%%%%%%%%%%%%%%%%%%%%%%%%%%%%%%%%%%%%%%%%%%%%%%%%%%%%%%%%%%%%%%%%%%%%%%
%%%%%%%%%%%%%%%%%%%%%%%%%%%%%%%%%%%%%%%%%%%%%%%%%%%%%%%%%%%%%%%%%%%%%%%%%%%%%%%%
% DISEÑO E IMPLEMENTACIÓN %
%%%%%%%%%%%%%%%%%%%%%%%%%%%%%%%%%%%%%%%%%%%%%%%%%%%%%%%%%%%%%%%%%%%%%%%%%%%%%%%%

\cleardoublepage
\chapter{Gamificar las Matemáticas}

\section{¿Se pueden gamificar las Matemáticas?}

Lo siguiente que necesitamos, si estamos convencidos de que la Gamificación puede ser una buena metodología, es tratar cómo se puede gamificar un aula.
%
Dedicaremos el próximo capítulo a ello.

\section{Estado de la Ley y del currículo}

Primero: ¿Permite o incluso anima la ley al empleo de esta metodología?

Segundo: Necesitamos elegir un curso y su currículo.

\subsection{Elección del curso para Gamificar}

El curso elegido es: porque...

\subsubsection{Contenidos de ese curso}
\subsubsection{Estándares de Aprendizaje Evaluables}


\section{Elementos de la gamificación aplicables al aula de Matemáticas}

\paragraph{Dinámicas}

\subparagraph{Narrativa}
\subparagraph{Progresión}

\paragraph{Mecánicas}

Las mecánicas que utilizaremos serán:
\subparagraph{Retos},
\subparagraph{Competición y cooperación},
\subparagraph{Feedback},
\subparagraph{Premios}

 Los premios serán: logros, medallas, puntos (que describiremos a continuación).

\paragraph{Elementos}

Los elementos que utilizaremos serán:
\subparagraph{Niveles},
\subparagraph{Logros},
\subparagraph{Puntos},
\subparagraph{Medallas},
\subparagraph{Jefes finales}...

\section{Programación anual Gamificada}

\subsection{Aprendizaje competencial}
\subsection{Cobertura de los contenidos}
\subsection{Incorporación de los Estándares de Aprendizaje Evaluables}

Matriz de trazabilidad entre los Estándares de Aprendizaje Evaluables y los componentes de la Gamificación:
\begin{table}[hbtp]
\caption{Matriz de trazabilidad entre los Estándares de Aprendizaje Evaluables y los componentes de la Gamificación.}
\label{Trazabilidad}
\begin{tabular}{c|c}
& Estándares de Aprendizaje Evaluables\\\hline
Elementos & \\\hline
\end{tabular}
\end{table}


\section{Evaluación de la gamificación}

Es fundamental saber si la gamificación tal como se ha llevado a cabo ha merecido la pena, qué aspectos son mejorables y cuáles hay que mantener.
%
Podríamos utilizar el marco propuesto por \cite{EvaluacionGamificacion} para autoevaluar la gamificación realizada.


%%%%%%%%%%%%%%%%%%%%%%%%%%%%%%%%%%%%%%%%%%%%%%%%%%%%%%%%%%%%%%%%%%%%%%%%%%%%%%%%
%%%%%%%%%%%%%%%%%%%%%%%%%%%%%%%%%%%%%%%%%%%%%%%%%%%%%%%%%%%%%%%%%%%%%%%%%%%%%%%%
% CONCLUSIONES %
%%%%%%%%%%%%%%%%%%%%%%%%%%%%%%%%%%%%%%%%%%%%%%%%%%%%%%%%%%%%%%%%%%%%%%%%%%%%%%%%

\cleardoublepage
\chapter{Conclusiones}
\label{chap:conclusiones}


%Se comenta pero no se pone como apartado específico: 
% \section{Trabajos futuros}

\label{sec:trabajos_futuros}

Puesta en marcha y evaluación. Detectar posibles mejoras.

Propuestas de Gamificación para otros cursos y otras asignaturas.



%%%%%%%%%%%%%%%%%%%%%%%%%%%%%%%%%%%%%%%%%%%%%%%%%%%%%%%%%%%%%%%%%%%%%%%%%%%%%%%%
%%%%%%%%%%%%%%%%%%%%%%%%%%%%%%%%%%%%%%%%%%%%%%%%%%%%%%%%%%%%%%%%%%%%%%%%%%%%%%%%
% APÉNDICE(S) %
%%%%%%%%%%%%%%%%%%%%%%%%%%%%%%%%%%%%%%%%%%%%%%%%%%%%%%%%%%%%%%%%%%%%%%%%%%%%%%%%

\cleardoublepage
\appendix


%%%%%%%%%%%%%%%%%%%%%%%%%%%%%%%%%%%%%%%%%%%%%%%%%%%%%%%%%%%%%%%%%%%%%%%%%%%%%%%%
%%%%%%%%%%%%%%%%%%%%%%%%%%%%%%%%%%%%%%%%%%%%%%%%%%%%%%%%%%%%%%%%%%%%%%%%%%%%%%%%
% BIBLIOGRAFIA %
%%%%%%%%%%%%%%%%%%%%%%%%%%%%%%%%%%%%%%%%%%%%%%%%%%%%%%%%%%%%%%%%%%%%%%%%%%%%%%%%

\cleardoublepage

% Las siguientes dos instrucciones es todo lo que necesitas
% para incluir las citas en la memoria
\bibliographystyle{apalike}
\bibliography{memoria}  % memoria.bib es el nombre del fichero que contiene
% las referencias bibliográficas. Abre ese fichero y mira el formato que tiene,
% que se conoce como BibTeX. Hay muchos sitios que exportan referencias en
% formato BibTeX. Prueba a buscar en http://scholar.google.com por referencias
% y verás que lo puedes hacer de manera sencilla.
% Más información: 
% http://texblog.org/2014/04/22/using-google-scholar-to-download-bibtex-citations/

\printindex

\end{document}
