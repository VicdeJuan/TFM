%!TEX root = ../TFM.tex
%\addcontentsline{toc}{chapter}{Resumen} % si queremos que aparezca en el ?dice
\markboth{RESUMEN}{RESUMEN} % encabezado


En el sistema educativo español es necesaria la mejora y la innovación. 
%
La última ley de educación, Ley Orgánica para la Mejora de la Calidad Educativa, lo tiene incluido en su propio nombre.

Aunque todavía no suficientes, cada vez son más los docentes y las instituciones que incorporan los nuevos avances de la investigación a sus prácticas docentes en las aulas.
%
Es importante apoyar esas iniciativas y dotarlas de recursos y posibilidades.
%
El presente trabajo trata precisamente eso, ofrecer un estudio detallado de la Gamificación como estrategia metodológica para el aula seguido de una propuesta concreta, diseñada y detallada lista para ser implementada directamente en el aula.


\begin{keywordsEs}
Gamificación, Taxonomía de Jugadores, Teorías de la Motivación y del Flow, Informes PISA , Unidad Didáctica, Matemáticas orientadas a las Ciencias Académicas, ESO, LOMCE
\end{keywordsEs}
