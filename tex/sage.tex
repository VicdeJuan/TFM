%!TEX root = ../TFM.tex

\defcitealias{Sage}{cocalc.com}

\section{Herramientas para la generación de retos}

Para la elaboración de los retos se han utilizado fundamentalmente 2 herramientas.
%
Para la parte gráfica, de diseño, formato, etc. se ha utilizado \url{draw.io}, una herramienta online para crear gráficos y diagramas.
%
Por otro lado, para preparar los ejercicios y asegurar que las cuentas son exactas y relativamente sencillas se ha utilizado \citetalias{Sage}, un simulador de Sage\footnote{Sage es un lenguaje de programación basado en Python pensado para resolver cálculos simbólicos.
%
Incluye muchas funciones para trabajar algebraicamente, por ejemplo, incorpora funcionalidad para trabajar con estructuras algebraicas como son los ideales de anillos de polinomios.
} online.

A continuación, incluimos un ejemplo del código utilizado. 
%
En concreto el código presentado genera los polinomios del último reto, tanto cooperativo como individual.
%
A partir de cuatro polinomios definidos al comienzo, realiza todos los cálculos requeridos, incluso genera el texto para ser incluido en los retos.

El código se puede probar online en \citetalias{Sage} \citep{Sage}.


\begin{python}
# Definicion del cuerpo de trabajo: Los racionales.
R.<x> = QQ[]

P1 = (x^2+1)*(x-2)
P2 = (x+1)*(x-3)
P3 = (x - 2) * x * (x + 2)
P4 = (x-2)*(x-3)

LP = [P1,P2,P3,P4]
numRetos=4

#### Definiciones auxiliares.
def _expand(P):
    _A = x^2+x+1
    if (type(P) == type(_A.factor())):
        return P.expand()
    else:
        return P

def _simplificable(P,Q):
    return P.gcd(Q).factor()

def _RootsTil(P,n):
    ret = []
    [ret.append(P.roots()[i][0]) for i in range(0,P.degree()-n)]
    return ret

def reto_indiv(_P,_Q,Contador):
    print "ESTE ES EL RETO ",Contador
    
    P = _expand(_P)
    Q = _expand(_Q)
    R = _expand(P.factor()[0][0]*Q.factor()[0][0])
    
    print "Tus 3 polinomios para trabajar son: P(x) = ",P," ; Q(x) = ",Q," y R(x) = ",R,".\n"
    
    M = P*Q
    print "Calcula: M(x) = P(x)Q(x) (sol: ",M,")."

    N = R*Q
    print "Calcula: N(x) = R(x)Q(x) (sol: ",N,")."

    D = N.factor()[1][0]*N.factor()[0][0]
    print "Calcula: D(x) = (",N.factor()[1][0],")(",N.factor()[0][0],")"

    R=N/D
    print "Calcula: R(x) = N(x) / D(x) (Sol: ",R,")\n Calcula: R(x) - (",R,") (Sol: ",R-R,")"

    
    print "Para terminar, factoriza M y N, sabiendo:"
    a = _RootsTil(M,3)
    if len(a) == 1:
        print "    ",a," es raiz de M"
    else:
        print "    ",a," son raices de M"
    
    a=_RootsTil(N,3)
    if len(a) == 1:
        print "    ",a," es raiz de N"
    else:
        print "    ",a," son raices de N"
    
    print "Sol: N(x) = ",N.factor()," y M(x) = ",M.factor(),""
        
    retval = _simplificable(M,N)
    print "Indica si hay algo que se pueda simplificar en M(x)/N(x) (Sol: ",retval,")"
    
    print "Escribe todos los terminos que puedas simplificar y llama $A_%d$ al resultado.\n\n\n\n\n"%Contador   
    return retval


#### Main

A = []
for i in xrange(numRetos): 
    A.append(reto_indiv(LP[i],LP[(i+1)%numRetos],i+1))
    

c=0
CS=[]
for i in xrange(len(A)):
    print "Poli: ",_expand(A[i])
    print "Divide por $D(x) = x^2+1$"
    print "Cociente: ",CS.append(_expand(A[i])//(x^2+1))
    f=_expand(A[i])//(x^2+1)
    print "Resto: ",_expand(A[i])%(x^2+1)
    c=c+f(6.177)
    print ""

print "Multiplica (x^2-1) y divide por x+1 y evalua en 1"
[(CS[i]*(x^2-1)/(x+1))(1) for i in range(len(CS))]

\end{python}

