\documentclass{article}

\usepackage{xspace}
\usepackage[utf8]{inputenc}
\usepackage{aurical}
\usepackage{pbsi}
\usepackage{lettrine}
\usepackage{soulutf8}

\setlength{\DefaultNindent}{1em}
\setcounter{DefaultLines}{1}
\input Romantik.fd
\newcommand*\initfamily{\usefont{U}{Romantik}{xl}{n}}
\renewcommand{\LettrineFontHook}{\initfamily}


\usepackage[spanish,es-tabla]{babel} % Comenta esta línea si tu memoria es en inglés
\usepackage[margin=1in]{geometry}


\DeclareRobustCommand{\augiefamily}{%
  \fontfamily{augie}\fontseries{m}\fontshape{n}\selectfont}
\DeclareTextFontCommand{\textaugie}{\augiefamily}


\renewcommand{\baselinestretch}{1.2}
\setlength{\parindent}{2em}
\setlength{\parskip}{4.5em}

% Paquetes adicionales

% --------------------

\renewcommand{\vec}[1]{\overrightarrow{#1}}

\begin{document}
\begin{table}
\centering
\caption{Resultado del desafío}
\begin{tabular}{|c|c|c|c|c|c|c|c|c|c|c|c|}
\hline
4 & 13 & 7 & 15 & 18 & 0 & 1 & 21 & 4 & 13 & -\\\hline
E & n  & h & o & r & a & b & u & e & n & -\\\hline
\end{tabular}

\small{Esta tabla no estará completa en el documento para estudiantes.}
\end{table}

\augiefamily{Querido diario},
\Fontauri

Después de un largo día de trabajo y reflexión, puedo sentarme a ordenar mis ideas.
%
Estoy muy emocinado con estas matemáticas que estoy desarrollando, ¡y estoy tan entusiasmado que te la voy a contar!.


Si a expresiones como $ax^n$ siendo $n$ un número entero, las llamamos monomios. ¿Cómo podríamos llamar a: $ax^2+bx+c$? 
%
Pues como son varios monomios juntos, ¡poli-nomio!
%
Y de la misma manera que $\frac{1}{x}$ no es un monomio, tampoco lo sería $\frac{x^2+3}{x}$, porque hay un monomio ($x$) que no está sumando ni restando, sino dividiendo.

\textbf{Un polinomio es, sencillamente una suma (¡o resta!) de monomios.}

Y por construirlo así, tenemos propiedades muy parecidas. ¿Te acuerdas del grado de un monomio?
%
Llamamos \textbf{grado de un polinomio} al mayor grado de sus monomios. Por ejemplo, el grado de $x^3+2x-1$ sería $3$.
%
¿Y el de $p(x) = x^4+2x-1$? ¿Y el de $q(x) = x+x^{13}-x^5$? ¿Y el de $\frac{2}{x}$?

Querido diario, espero que te hayas dado cuenta de la trampa en el último y no te lo tomes a mal jeje.
%
Arriba del todo hay una tabla para ir apuntando los valores que te salen. ¡Apunta en las 2 primeras casillas el grado de los 2 primeros polinomios!
%
Ya que estamos... ¿y el grado de $3x^7+5x^2+5x^3+5x^7-4x^3+5x^0$? ¡Apúntalo también!

La utilidad de los monomios era poder traducirlos a un número y decir, si $x=2$, ¿cuánto valdría $x^3+2x$? 
%
Esto es el \textbf{valor numérico}.
%
Como un polinomio simplemente es la suma de monomios, también podemos hacer ¡el valor numérico de un polinomio!. 
%
Mira, atento:
%
Tenemos $p(x) = 5x^3+1$ y queremos calcular $p(2)$, es decir, el valor numérico de $p(x)$ para $x=2$, sustituimos la $x$ por un $2$ y ¡operamos! $p(2) = 2^3+1=8+1=9$.
%
¿Y cuál sería el valor numérico\footnote{Apunta el resultado en la casilla correspondiente de la tabla inicial.} de $p_2(x) = x^4-1$ para $x=2$? 



¿Y se pueden tener varias \textit{letras} en el polinomio? ¡Claro! ¿Cuál sería el valor numérico\footnotemark[1] de $p_3(x,y) = x^3+y^5-3xy^2-1$ para $x=3,y=1$?




Y ahora,  todo esto se puede combinar y recombinar y alcombinar.
%
Por ejemplo, si tenemos $q_1(x) = x^4-5x^2+6x$ y $q_2(x) = x^3-5x^+6$, ¿cuál sería el valor numérico\footnotemark[1] de $p(x) = q_1(x) - x\cdot q_2(x)$ para $x=2$, es decir, $p(x) = (x^4-5x^2+6x) - x\cdot (x^3-5x+6)$ para $x=2$?



\quad

¿Serías capaz de calcular\footnotemark[1] $p(1)$ siendo $p(x) = 5 x^{150} + 7x^{75} - 9x-2x^{x}$? ¡Y fíjate que ni siquiera es un polinomio! Pero así de versátiles son las matemáticas.



Un último truco, ¡que me quedo sin papel y ya estás cerca de descifrar el mensaje!
%
Piensa en $q_1(x,y) = xy^2-6xy$ y calcula su valor numérico\footnotemark[1] para $x=3,y=7$.



Mira con atención $q_1(x,y) = xy^2-6xy$. Hay 2 $x$ que están multiplicando, ¿verdad? ¿Podríamos \textit{sacarlas} y escribirlo de otra manera?
%
Aquí esta el truquillo del \textbf{Factor Común}:
%
$q_1(x,y) = xy^2-6xy = xy^2-6xy = x\cdot (y^2-6y)$. 
%
Calcula ahora el valor numérico.
%
¿Más sencillo?

Ahora tu, para obtener el siguiente número necesario para descifrar el mensaje, ¿cuál es el mayor grado\footnotemark[1] del factor común que podemos sacar en
$q_2 = x^4y^3z + x^4y^2 + x^5y^6$?



Ahora, saca todos los factores comunes que puedas de $q_2$
% Sol: x^4y^2(yz+1+xy^4) 
y suma todos los exponentes de toda parte literal. ¡Ese número ocupa la penúltima posición del mensaje!


Tal vez seas capaz de descifrar ya el mensaje, pero sino, ¿Cuál es el valor numérico \footnotemark[1] de $p(x,y) = x^3y^5 + xy^4+x^2y^3+5xy+13y^2x$ para $x=0,y=13$?


%% Apendices (ejercicios, examenes)


\end{document}
