\chapter{Educación gamificada}

Hasta ahora la Gamificación ha sido tratada en términos generales, sin centrarnos específicamente en la educación. 
%
A continuación, se estudia cómo se adapta esa teoría general al ecosistema de un aula.


\coment{Porqué gamificar la educación}

\cite{lee2011gamification} sugieren que el sistema educativo ya contiene elementos de la gamificación, más concretamente \gls{PBL}, ya que los estudiantes realizan unos exámenes para obtener una calificación (puntos) y si se han obtenido más de 5 puntos, se obtiene la medalla de \textit{aprobado}.
%
Incluso, se puede obtener la medalla de \textit{mención de honor} en la evaluación final.
%
Además, encubiertamente se forma un \textit{leaderboard}, ya que los alumnos tienen claro quienes son los mejores y los peores alumnos (en términos de calificaciones obtenidas).
%
Sin embargo, estos elementos no implican necesariamente motivación por parte de los alumnos.
%
Diseñar una gamificación en un aula puede motivar a los estudiantes, ofrecer a los profesores mejores herramientas para guiar y recompensar a sus estudiantes y enseñar a los estudiantes que la educación puede ser una experiencia divertida  \citep{lee2011gamification}.

Es importante tener en cuenta la teoría de la auto-determinación (ver \ref{SDT}) y diseñar una gamificación en el contexto educativo que no debe basarse en una motivación extrínseca, sino fomentar la motivación intrínseca. 
%
De esta manera, los estudiantes podrán desarrollar una mayor capacidad para valorar el aprendizaje y cuando avancen a lo largo del sistema educativo sean capaces de aprovechar contextos educativos no gamificados. 
%
De esta manera, nos interesaría explorar la ruta de evolución \citep{marczewski}  para convertir jugadores extrínsecamente motivados en jugadores intrínsecamente motivados (ver \ref{fig::MarczewskiEvol})


En la sección \ref{sec:EstadoEducacionMates} se comentó la existencia de un círculo vicioso (ver figura \ref{fig::circuloVicioso}) y que la variable más influyente en el rechazo hacia las Matemáticas en España es la percepción de la materia como aburrida o divertida.
%
Esperamos que, tras lo expuesto, el lector esté de acuerdo en que la Gamificación puede ayudar a modificar esa percepción aprovechando que existen tareas difíciles y divertidas (ver \ref{kindsoffun}) además de romper el círculo vicioso por 2 extremos diametralmente opuestos: aburrimiento y desmotivación.

\coment{¿Se puede gamificar la educación? Sí, Cook. 2013}


\section{Proceso de diseño de una Gamificación}

Para diseñar una buena gamificación es necesario seguir un proceso y hay quienes han propuesto un marco con unas pautas para seguir en la tarea. 
%
Como no hay un método consensuado presentaremos 2: uno más general  \citep{werbach2012win} y otro más aplicado al contexto educativo  \citep*{kapp2013gamification}.

Werbach define 6 pasos para diseñar una buena gamificación con 6 D's: 
1 - Definir los objetivos de negocio; 2 - Delinear los comportamientos deseados; 3 - Describir a los jugadores; 4 - Diseñar los bucles de actividades; 5 - No olvidarse de la diversión (en inglés: \textit{Don't forget the fun}); 6 - Implementar las herramientas apropiadas (en inglés: \textit{Deploy}).
%
Esta propuesta es demasiado general y no tiene en cuenta algunas características fundamentales del contexto educativo, por ejemplo, la necesidad de un sistema de evaluación.
%
Por ello, una propuesta más centrada en el ámbito educativo puede resultar más útil, sin ignorar por completo la propuesta de Werbach.

La otra propuesta,  \cite{kapp2013gamification}, tiene algunos elementos en común con la de Werbach, pero otros diferentes. 
%
Los autores establecen que se debe pasar por cuatro fases en la gamificación de un contexto educativo. 1 - Responder a las preguntas base; 2 - Responder a las preguntas de práctica; 3 - Diseñar el sistema de valoración y clasificación; 4 - Jugar al juego.

Las preguntas base hacen referencia a 5 aspectos: identificar el problema, estudiar los comportamientos existentes, definir los comportamientos deseados, tener claro el objetivo competencial (competencias que los estudiantes necesitan adquirir para que consideremos éxitosa la gamificación) y valorar aspectos que pueden mostrarnos que los alumnos están aprendiendo.
%
Por otro lado, las preguntas de práctica son las preguntas sobre el público objetivo de la gamificación (edad, conocimientos previos, habilidades, tipos de jugadores, etc.) , la logística (lugar, momento, tiempo invertido y dinámicas, mecánicas y componentes a utilizar) y las cuestiones técnicas (la disponibilidad de herramientas TIC o no, tanto en el contexto escolar como en el contexto familiar de los estudiantes).
%
En cuanto al sistema de valoración y clasificación es necesario un arduo trabajo.
%
La base logística del sistema tiene que ser completa, es decir, no puede darse el caso en el que no esté especificada la obtención o no de una recompensa o la siguiente meta a alcanzar.
%
Este aspecto es importante, ya que puede haber jugadores que se dediquen a buscar fallos en el sistema y a romperlo para ganar.
%
Por ejemplo, los jugadores de tipo perturbador, según la teoría de  \citet{marczewski}, más concretamente los duelistas dentro de los perturbadores podrían romper la gamificación si encontraran una incompletitud o inconsistencia en el sistema evaluativo.
%
Además, es necesario que el sistema sea justo y permita obtener calificaciones acordes con las competencias y habilidades adquiridas, por ello, las actividades del proyecto, la valoración y el resultado final deben ir unidos.
%
Por último, hay que saber qué acciones pueden realizar los jugadores cuando interactúen (distribuir recursos, coleccionar, cooperar, realizar misiones por otros jugadores, reintentar tareas, etc.). 
%
Es necesario también definir los estados ganadores y el número de oportunidades en cada actividad.

Durante la implementación de la gamificación es importante no perder de vista uno de los puntos que incluye Werbach: No olvidar la diversión.
%
Si un aula gamificada no es divertida para los estudiantes necesitamos revisar su diseño y reoformarla.
%
El comienzo del siguiente capítulo tendrá como marco de referencia los pasos que acabamos de tratar para guiar el proceso de diseño de la gamificación para un aula de Matemáticas en la Educación Secundaria del sistema educativo español.


\section{Estado de la Gamificación en la educación en España}



\section{La Gamificación en la enseñanza de Matemáticas}
