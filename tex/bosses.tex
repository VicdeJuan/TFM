%!TEX root = ../TFM.tex


Todos los retos siguen el mismo esquema, cambiando los polinomios iniciales y, por tanto, los resultados intermedios y el resultado final.
%
Cada alumno obtendrá como resultado un polinomio $A_i(x)$ necesario para el reto cooperativo.

\paragraph{Alumno 1:\\}



Tus 3 polinomios para trabajar son: $P(x) =  x^3 - 2x^2 + x - 2 $ ; $Q(x) =  x^2 - 2x - 3 $ y $R(x) =  x^2 - 5x + 6 $.



Calcula: $M(x) = P(x)\cdot  Q(x)$ (sol: $ x^5 - 4x^4 + 2x^3 + 2x^2 + x + 6 $).

Calcula: $N(x) = R(x)\cdot  Q(x)$ (sol: $ x^4 - 7x^3 + 13x^2 + 3x - 18 $).

Calcula: $D(x) = ( x + 1 )\cdot  ( x - 2 )$

Calcula: $R(x) = N(x) / D(x)$ (Solución: $ x^2 - 6x + 9 $)

Calcula: $R(x) - ( x^2 - 6x + 9 )$ (Solución: $ 0 $)

Para terminar, factoriza $M$ y $N$, sabiendo:

     [3, 2]  son raíces de $M$

     [2]  es raíz de $N$

Solución: $N(x) =  (x - 2) \cdot  (x + 1) \cdot  (x - 3)^2 $ y $M(x) =  (x - 3) \cdot  (x - 2) \cdot  (x + 1) \cdot  (x^2 + 1) $

Toma \[\frac{M}{N} = \frac{\quad\quad\quad\quad}{\quad\quad\quad\quad}\]

¿Cuáles se pueden simplificar? $\left[\text{Solución: } (x - 3) \cdot  (x - 2) \cdot  (x + 1) \right]$

Escribe todos los términos que puedas simplificar y llama $A_1(x)$ al resultado.













\paragraph{Alumno 2:\\}



Tus 3 polinomios para trabajar son: $P(x) =  x^2 - 2x - 3 $ ; $Q(x) =  x^3 - 4x $ y $R(x) =  x^2 - 5x + 6 $.



Calcula: $M(x) = P(x)\cdot  Q(x)$ (sol: $ x^5 - 2x^4 - 7x^3 + 8x^2 + 12x $).

Calcula: $N(x) = R(x)\cdot  Q(x)$ (sol: $ x^5 - 5x^4 + 2x^3 + 20x^2 - 24x $).

Calcula: $D(x) = ( x )\cdot  ( x - 3 )$

Calcula: $R(x) = N(x) / D(x)$ (Solución: $ x^3 - 2x^2 - 4x + 8 $)

Calcula: $R(x) - ( x^3 - 2x^2 - 4x + 8 )$ (Solución: $ 0 $)

Para terminar, factoriza $M$ y $N$, sabiendo:

     [3, 2]  son raíces de $M$

     [3, 0]  son raíces de $N$

Solución: $N(x) =  (x - 3) \cdot  x \cdot  (x + 2) \cdot  (x - 2)^2 $ y $M(x) =  (x - 3) \cdot  (x - 2) \cdot  x \cdot  (x + 1) \cdot  (x + 2) $

Toma \[\frac{M}{N} = \frac{\quad\quad\quad\quad}{\quad\quad\quad\quad}\]

¿Cuáles se pueden simplificar? $\left[\text{Solución: } (x - 3) \cdot  (x - 2) \cdot  x \cdot  (x + 2) \right]$

Escribe todos los términos que puedas simplificar y llama $A_2(x)$ al resultado.













\paragraph{Alumno 3:\\}



Tus 3 polinomios para trabajar son: $P(x) =  x^3 - 4x $ ; $Q(x) =  x^2 - 5x + 6 $ y $R(x) =  x^2 - 5x + 6 $.



Calcula: $M(x) = P(x)\cdot  Q(x)$ (sol: $ x^5 - 5x^4 + 2x^3 + 20x^2 - 24x $).

Calcula: $N(x) = R(x)\cdot  Q(x)$ (sol: $ x^4 - 10x^3 + 37x^2 - 60x + 36 $).

Calcula: $D(x) = ( x - 2 )\cdot  ( x - 3 )$

Calcula: $R(x) = N(x) / D(x)$ (Solución: $ x^2 - 5x + 6 $)

Calcula: $R(x) - ( x^2 - 5x + 6 )$ (Solución: $ 0 $)

Para terminar, factoriza $M$ y $N$, sabiendo:

     [3, 0]  son raíces de $M$

     [3]  es raíz de $N$

Solución: $N(x) =  (x - 3)^2 \cdot  (x - 2)^2 $ y $M(x) =  (x - 3) \cdot  x \cdot  (x + 2) \cdot  (x - 2)^2 $

Toma \[\frac{M}{N} = \frac{\quad\quad\quad\quad}{\quad\quad\quad\quad}\]

¿Cuáles se pueden simplificar? $\left[\text{Solución: } (x - 3) \cdot  (x - 2)^2 \right]$

Escribe todos los términos que puedas simplificar y llama $A_3(x)$ al resultado.













\paragraph{Alumno 4:\\}



Tus 3 polinomios para trabajar son: $P(x) =  x^2 - 5x + 6 $ ; $Q(x) =  x^3 - 2x^2 + x - 2 $ y $R(x) =  x^2 - 5x + 6 $.



Calcula: $M(x) = P(x)\cdot  Q(x)$ (sol: $ x^5 - 7x^4 + 17x^3 - 19x^2 + 16x - 12 $).

Calcula: $N(x) = R(x)\cdot  Q(x)$ (sol: $ x^5 - 7x^4 + 17x^3 - 19x^2 + 16x - 12 $).

Calcula: $D(x) = ( x - 2 )\cdot  ( x - 3 )$

Calcula: $R(x) = N(x) / D(x)$ (Solución: $ x^3 - 2x^2 + x - 2 $)

Calcula: $R(x) - ( x^3 - 2x^2 + x - 2 )$ (Solución: $ 0 $)

Para terminar, factoriza $M$ y $N$, sabiendo:

     [3, 2]  son raíces de $M$

     [3, 2]  son raíces de $N$

Solución: $N(x) =  (x - 3) \cdot  (x - 2)^2 \cdot  (x^2 + 1) $ y $M(x) =  (x - 3) \cdot  (x - 2)^2 \cdot  (x^2 + 1) $

Toma \[\frac{M}{N} = \frac{\quad\quad\quad\quad}{\quad\quad\quad\quad}\]

¿Cuáles se pueden simplificar? $\left[\text{Solución: } (x - 3) \cdot  (x - 2)^2 \cdot  (x^2 + 1) \right]$

Escribe todos los términos que puedas simplificar y llama $A_4(x)$ al resultado.





