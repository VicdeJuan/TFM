%!TEX root = ../TFM.tex


\chapter{Educación gamificada}

A continuación, se estudia cómo se adapta toda la teoría expuesta al ecosistema de un aula.

\coment{Porqué gamificar la educación}

\cite{lee2011gamification} sugieren que el sistema educativo ya contiene elementos de la gamificación, más concretamente \gls{PBL}, ya que los estudiantes realizan unos exámenes para obtener una calificación (puntos) y si se han obtenido más de 5 puntos, se obtiene la medalla de \textit{aprobado}.
%
Incluso, se puede obtener la medalla de \textit{mención de honor} en la evaluación final.
%
Además, encubiertamente se forma un \textit{leaderboard}, ya que los alumnos tienen claro quiénes son los mejores y los peores alumnos (en términos de calificaciones obtenidas).
%
Sin embargo, estos elementos no implican necesariamente motivación por parte de los alumnos.
%
Diseñar una gamificación en un aula puede motivar a los estudiantes, ofrecer a los profesores mejores herramientas para guiar y recompensar a sus estudiantes y enseñar a los estudiantes que la educación puede ser una experiencia divertida  \citep{lee2011gamification}.

Es importante tener en cuenta la teoría de la auto-determinación \crossref{(ver \ref{SDT})} y diseñar una gamificación en el contexto educativo que no debe basarse en una motivación extrínseca, sino fomentar la motivación intrínseca. 
%
De esta manera, los estudiantes podrán desarrollar una mayor capacidad para valorar el aprendizaje y cuando avancen a lo largo del sistema educativo sean capaces de aprovechar contextos educativos no gamificados. 
%
De esta manera, nos interesaría explorar la ruta de evolución \citep{marczewski}  para convertir jugadores extrínsecamente motivados en jugadores intrínsecamente motivados (ver \ref{fig::MarczewskiEvol})


Se ha comentado la existencia de un círculo vicioso (ver figura \ref{fig::circuloVicioso}) y que la variable más influyente en el rechazo hacia las Matemáticas en España es la percepción de la materia como aburrida o divertida.
%
Sería de esperar que, tras lo expuesto, el lector esté de acuerdo en que la Gamificación puede ayudar a modificar esa percepción aprovechando que existen tareas difíciles y divertidas \crossref{(ver \ref{kindsoffun})} además de romper el círculo vicioso por 2 extremos diametralmente opuestos: aburrimiento y desmotivación.

\coment{¿Se puede gamificar la educación? Sí, Cook. 2013}


\section{Proceso de diseño de una Gamificación}

Para diseñar una buena gamificación es necesario seguir un proceso y hay quienes han propuesto un marco con unas pautas para seguir en la tarea. 
%
Como no hay un método consensuado se presentan 2: uno más general  \citep{werbach2012win} y otro más aplicado al contexto educativo  \citep*{kapp2013gamification}.

Werbach define 6 pasos para diseñar una buena gamificación con 6 D's: 
1 - Definir los objetivos de negocio; 2 - Delinear los comportamientos deseados; 3 - Describir a los jugadores; 4 - Diseñar los bucles de actividades; 5 - No olvidarse de la diversión (en inglés: \textit{Don't forget the fun}); 6 - Implementar las herramientas apropiadas (en inglés: \textit{Deploy}).
%
Esta propuesta es demasiado general y no tiene en cuenta algunas características fundamentales del contexto educativo, por ejemplo, la necesidad de un sistema de evaluación.
%
Por ello, una propuesta más centrada en el ámbito educativo puede resultar más útil, sin ignorar por completo la propuesta de Werbach.

La otra propuesta,  \cite{kapp2013gamification}, tiene algunos elementos en común con la de Werbach, pero otros diferentes. 
%
Los autores establecen que se debe pasar por cuatro fases en la gamificación de un contexto educativo. 1 - Responder a las preguntas base; 2 - Responder a las preguntas de práctica; 3 - Diseñar el sistema de valoración y clasificación; 4 - Jugar al juego.

\label{PasosGamificar}
%
Las preguntas base hacen referencia a 5 aspectos: identificar el problema, estudiar los comportamientos existentes, definir los comportamientos deseados, tener claro el objetivo competencial (competencias que los estudiantes necesitan adquirir para que se considerara exitosa la gamificación) y valorar aspectos que pueden mostrarnos que los alumnos están aprendiendo.
%
Por otro lado, las preguntas de práctica son las preguntas sobre el público objetivo de la gamificación (edad, conocimientos previos, habilidades, tipos de jugadores, etc.), la logística (lugar, momento, tiempo invertido y dinámicas, mecánicas y componentes a utilizar) y las cuestiones técnicas (la disponibilidad de herramientas TIC o no, tanto en el contexto escolar como en el contexto familiar de los estudiantes).
%
En cuanto al sistema de valoración y clasificación es necesario un arduo trabajo.
%
La base logística del sistema tiene que ser completa, es decir, no puede darse el caso en el que no esté especificada la obtención o no de una recompensa o la siguiente meta a alcanzar.
%
Este aspecto es importante, ya que puede haber jugadores que se dediquen a buscar fallos en el sistema y a romperlo para ganar.
%
Por ejemplo, los jugadores de tipo perturbador, según la teoría de  \citet{marczewski}, más concretamente los duelistas dentro de los perturbadores podrían romper la gamificación si encontraran una incompletitud o inconsistencia en el sistema evaluativo.
%
Además, es necesario que el sistema sea justo y permita obtener calificaciones acordes con las competencias y habilidades adquiridas, por ello, las actividades del proyecto, la valoración y el resultado final deben ir unidos.
%
Por último, hay que saber qué acciones pueden realizar los jugadores cuando interactúen (distribuir recursos, coleccionar, cooperar, realizar misiones por otros jugadores, reintentar tareas, etc.). 
%
Es necesario también definir los estados ganadores y el número de oportunidades en cada actividad.

Durante la implementación de la gamificación es importante no perder de vista uno de los puntos que incluye Werbach: No olvidar la diversión.
%
Si un aula gamificada no es divertida para los estudiantes será necesario revisar su diseño y reformarla.


\section{Estado de la Gamificación en la educación en España}

La Gamificación como metodología docente está en auge. 
%
Una prueba de ello es el \gls{MOOC} del \gls{INTEF} que tuvo su primera edición en octubre de 2016.
%
Otra prueba es la iniciativa: \textit{Gamifica tu aula}, un grupo de docentes en España desde infantil hasta el ámbito universitario que emplean esta metodología y utilizan una web \footnote{\url{http://gamificatuaula.wixsite.com/ahora}} con para darse a conocer, compartir recursos y ayudar a docentes que se quieran iniciar en el campo de la gamificación.
%
Tienen también un perfil en twitter\footnote{\href{https://twitter.com/gamificatuaula}{@gamificatuaula}} con casi 3000 seguidores, con el que intentan dar difusión a sus propuestas y ser contactados.
%
Entre estos profesores se encuentra Javier Espinosa Gallardo, Premio Innovación SIMO 2015 y Premio Nacional de Educación 2015, ambos premios por su proyecto de gamificación interdisciplinar \href{http://jespinosag.wixsite.com/classofclans}{\textit{Class of clans}}.

Al presentarse en la web, dicen:
%
\comillas{Tenemos la clara convicción, a partir de experimentar en nuestras aulas, de que [la Gamificación] mejora los procesos de aprendizaje, la motivación, el desarrollo de la inteligencia emocional y la adquisición de habilidades como la cooperación o la resiliencia, entre otras.}
%
y esta convicción quieren difundirla y \comillas{colonizar}.
%
Consideran además que \comillas{A gamificar se aprende gamificando}.

Se observa que no es novedad absoluta plantear la Gamificación en la educación España, pues ya se está utilizando.
%
Sin embargo, la tasa adopción por parte de los docentes es escasa, ya que el número de centros y de profesores es inmensamente mayor que el número de profesores empleando la Gamificación.
%
En parte puede deberse al desconocimiento por parte de los docentes de esta metodología o al sobresfuerzo que puede ser necesario para diseñar una buena Gamificación.

\section{La Gamificación en la enseñanza de Matemáticas}

Lo ideal sería poder diseñar gamificaciones interdisciplinares por varias razones.
%
La primera y más importante, por la necesidad y beneficios de un sistema educativo menos compartimentado. 
%
Los problemas globales han aumentado en complejidad y conectividad (crisis de refugiados, del agua, cambio climático, crecimiento poblacional, etc.), lo que obliga a enfocarlos como complejos, inseparables y retroalimentados, desde una perspectiva interdisciplinar \citep{Interdiscip}
%
Por otro lado, la duplicidad de esfuerzos que supondría realizar gamificaciones distintas para aulas distintas sería un disparate.

Sin embargo, en la realidad los escenarios ideales no se alcanzan a la primera.
%
Además, las experiencias de gamificación interdisciplinar según \textit{gamificatuaula.com} han surgido de una gamificación monodisciplinar que ha ido contagiando e involucrando a otros docentes.


Por otro lado, no todas las mecánicas, dinámicas y componentes de la Gamificación se pueden aplicar por igual a las diferentes asignaturas.
%
Por ejemplo, \cite{ClassAVideogame} diseñó una gamificación en su aula de biología en la que los distintos niveles eran distintas fases en la investigación de un virus para sobrevivir una pandemia.
%
Esos niveles se ajustan relativamente bien al curriculum de la asignatura y al temario a impartir.
%
Otro ejemplo sería una gamificación diseñada por Javier Espinosa, en la que los distintos niveles eran especies animales, de menor a mayor complejidad.
%
En asignatura como Historia y Geografía, los niveles se pueden diseñar como ascensos dentro de un sistema feudal \citep{Feudal}.

En Matemáticas esa herramienta es difícil de aplicar con un sentido tan pleno.
%
Sin embargo, hay otros elementos que, por las características propias de cada asignatura, son muy útiles.
%
Las Matemáticas permiten plantear retos (ejercicios y problemas) de muy diversa dificultad: se puede plantear un reto de comienzo de la clase, el reto del día, el reto de la semana, etc. con la dificultad que en cada momento sea necesaria.
%
Además, se pueden plantear problemas contextualizados en la narración de la gamificación, por ejemplo, de geometría.
%
Si se utilizara una narrativa sobre la sociedad de la Grecia antigua, como el proyecto Helade (2015/2016), los problemas a los que se enfrentaban Pitágoras y Tales se pueden contextualizar en esa narrativa a la perfección.


Independientemente de la narrativa que se utilice hay una gran cantidad de elementos disponibles a utilizar en la gamificación. 
%
Por las características de las Matemáticas algunos elementos que podrían tener más sentido serían las misiones y los retos, los jefes finales y el desbloqueo de herramientas, pudiendo desbloquear teoremas y resultados para resolver problemas más difíciles.
