
\section{La Gamificación}

\coment{Esto es lo que correspondería al marco teórico.}


En esta sección procedemos a describir en términos generales la Gamificación como estrategia metodológica.
%
Primeramente trataremos de clarificar el concepto de gamificación, porqué y cómo se puede gamificar un contexto.
%
Una vez detallado, en la siguiente sección trataremos de aplicar la gamificación en la educación.


La palabra \textit{Gamificación} es una traducción del término inglés \textit{Gamification}, palabra derivada del sustantivo \textit{game}.
%
En castellano no podemos considerar Gamificación como palabra derivada de otro sustantivo. 
%
Otra posible traducción sería ludificación, palabra derivada del adjetivo lúdico.
%
En este sentido podríamos utilizar la denominación ludificación, pero se ha preferido utilizar en esta tesis el término Gamificación para buscar la congruencia con la tendencia entre el profesorado.


\coment{Qué es la Gamificación}


Atendiendo a \cite{GamificationDef} podemos definir \concept{Gamificación} como \textit{el uso de elementos de juegos en contextos no lúdicos}. 

Esta definición no está restringida al ámbito educativo. 
%
De hecho, la Gamificación entendida como una estrategia o metodología es aplicada a día de hoy en diversos ámbitos. 
%
Por ejemplo,es una técnica muy utilizada en el campo del Marketing y de los Recursos Humanos.
%
Tanto la educación como el Marketing son contextos no lúdicos en los que al introducir de elementos de juegos estaríamos gamificando. 
%
Pero, ¿qué son elementos de juegos? 
%
Aquellas dinámicas, mecánicas y componentes que hacen atractivos los juegos.
%
Es importante clarificar que el término juegos engloba tanto juegos de mesa como videojuegos y que no todos los juegos tienen los mismos elementos.
%
Algunos juegos, por sus características, utilizan más unos determinados elementos que otros.
%
Con fines didácticos vamos a describir algunos de esos elementos, aunque más adelante se ofrecerá una lista más completa.

Por ejemplo, en los juegos se produce un feedback instantáneo: el jugador avanza un número de casillas, obtiene dinero al pasar por una casilla determinada, aparecen carteles con información sobre el desempeño (¡Sigue así!), etc.
%
Este feedback instantáneo no se produce en contextos no lúdicos.
%
Un trabajador de una empresa es evaluado (como mucho) una vez al año. 
%
¿Y si cada día recibiera un pequeño feedback sobre su rendimiento?


Otro elemento importante de algunos juegos es la posibilidad de tomar decisiones importantes.
%
De hecho, algunos juegos van modificando la historia del juego a medida que el jugador avanza. 
%
Esto ocurre en videojuegos (la saga Mass Effect sería un ejemplo) pero también en los tradicionales juegos de rol (Dragones y Mazmorras).
%
¿Trabajarán mejor los empleados de una empresa si dicha empresa si pueden elegir en qué trabajar?
%
Algunos grandes proyectos de Google han surgido como resultado del porcentaje de tiempo que la empresa establece para que sus trabajadores trabajen en sus propios proyectos.

Un consumidor puede elegir entre 2 fruterías con la misma variedad de frutas y precios similares.
%
Si en una de ellas cada semana hay una nueva oferta para un tipo de fruta específico y es la fruta de la semana podría despertar atracción en el consumidor, incluso, que el consumidor quiera que empiece una nueva semana para ver cuál es la nueva oferta de la frutería.
%
A este fenómeno se le denomina \concept{bucle de atracción}. 
%
Es muy típico en los videojuegos online que existan misiones diarias, misiones semanales e incluso eventos especiales en fechas especiales como podría ser Navidad o Primavera.
%
¿Mejoraría la motivación de los estudiantes de una asignatura que el profesor planteara un reto opcional cada semana?


\coment{Por qué gamificar}

\paragraph{¿Por qué gamificar?} Se han dejado algunas preguntas en el aire, pero todas ellas se pueden englobar en la siguiente pregunta: ¿La Gamificación en un determinado contexto concreto mejora el rendimiento de las personas en ese contexto?
%
Hay bastantes investigaciones recientes que intentan contestar a esta pregunta.
%
Recurriendo a una revisión bibliográfica \cite{EmpiricalGamification} encontramos que la mayoría de experimentos empíricos sobre Gamificación han tenido efectos positivos en términos motivacionales.
%
Sin embargo, no todos los estudios encontraron efectos positivos en todos los participantes.
%
Además, parece que la gamificación falla a largo plazo, tal vez por el efecto de la novedad. 
%
\footnote{Estos posibles peligros y otros se tratarán más adecuadamente en \ref{PosiblesPeligros}, cuando el lector disponga de visión más global de la gamificación.}

Pero la gamificación puede producir efectos beneficiosos más allá de la motivación.
% 
De acuerdo con el profesor Kevin Werbach de la Universidad de Pensilvania la Gamificación permite fidelizar a las persona, hacer todavía más social el contexto, ofrecer a la persona un sentido del progreso en ese contexto y crear un hábito.


\subsection{Diferencias entre Juegos serios y Gamificación}

Es importante distinguir gamificación de juegos serios. Los \concept[Juegos serios]{juegos serios} consisten en la modificación del contexto transformándolo en un contexto lúdico, mientras que la gamificación incorpora elementos en un contexto no lúdico, manteniendo el contexto como no lúdico.
%
Un ejemplo de juego serio sería idear un juego de conquistas como el Risk para trabajar los mapas políticos con los alumnos. 

Aunque los juegos serios tengan consecuencias positivas y puedan ser una buena herramienta\footnote{Tanto es así que \cite{MetaSerious} concluye en su revisión que los juegos serios son más efectivos en contextos de aprendizaje pero menos efectivo que los métodos convencionales en términos motivacionales.}, su estudio se sale de esta tesis.

\subsection{Aspectos de la Gamificación}

\paragraph{Diseño enfocado en la persona:} 
Un contexto gamificado crea una experiencia.
%
La gamificación es \textit{human-focused design} en contraposición con \textit{function-focused design} \cite{BeyondPBL}.
%
La situación se transforma en una experiencia para el usuario, y esta es una de las claves.

Aunque pueda ser una obviedad, en un juego hay jugadores. No son tratados como participantes ni como usuarios, sino como jugadores.
%
Pensar en las personas participantes de un contexto que se quiere gamificar como jugadores es un paso importante.
%
Sitúa a esas personas como los protagonistas y como el centro de la experiencia, pues para eso están diseñados los juegos.
%
Además, los jugadores tienen un cierto sentido de autonomía y control sobre la experiencia.
%
Otra clave a considerar es la siguiente: un juego tiene la meta de que sus jugadores empiecen a jugar, frente a otras posibilidades a su alcance, y se mantengan jugando.
%
Una manera de conseguir esta meta es hacer el recorrido del juego o de la experiencia gamificada con una dificultad que se incremente progresivamente y de acuerdo a lo que el jugador puede conseguir en cada momento. 
%
Según el profesor Kevin Werbach, es importante que al principio del recorrido sea imposible fracasar, para lo que será necesario diseñar guías y limitar las posibilidades existentes e ir desbloqueándolas a medida que el jugador avanza en la experiencia.



\paragraph{La anatomía de la diversión: }

Una herramienta muy importante mediante la cual los juegos consiguen las claves mencionadas anteriormente es mediante la diversión.
%
Es impensable un juego que sea aburrido de jugar.
%
De la misma manera, una gamificación tiene que ser divertida.
%
Es importante esclarecer que hay varios tipos de diversión.
%
De acuerdo con \cite{whyweplaygames} hay 4 tipos de diversión: 
%
\concept[Diversión\IS Fácil]{Diversión fácil} -- aquella diversión que se produce ante un disfrute de la experiencia, manteniendo la atención del jugador. Se basa en la curiosidad y la intriga --;
%
\concept[Diversión\IS Difícil]{Diversión difícil}  -- aquella diversión que producen los retos que requieren habilidad y estrategia más que suerte y que permiten al jugador constatar cuan bueno es--;
%
\concept[Diversión\IS Social]{Diversión social} -- aquella diversión que se fundamenta en la relación con otras personas durante el juego --;
%
\concept[Diversión\IS Interna]{Diversión interna} -- aquella diversión producida por las experiencias internas como el alivio, el entusiasmo y la agitación.

Esta no es la única clasificación de la diversión.
%
En \cite{MDA} encontramos una división en 8 tipos: sensación, fantasía, narrativa, retos, social, descubrimiento, expresión, pasatiempo.
%
\label{AnatomyOfFun}
%

Esta anatomía de la diversión es fundamental para la gamificación.
%
Un entorno laboral o un contexto educativo normalmente no se caracterizan por ser intrínsecamente divertidos.
%
¿Podemos enfocar la gamificación para trabajar o aprender desde alguna de las dimensiones de la diversión? 
%
El potencial de la diversión difícil es inmenso. 
%
De hecho, algunas personas buscan que su trabajo sea un reto continuo.
%
Por otro lado, ¿cuántos estudiantes van a los centros educativos porque van sus amigos?
%
Y eso podría considerarse que la dimensión social les divierte.



\subsubsection{Elementos de la Gamificación}
Los elementos de la Gamificación son los elementos que podemos encontrar en los juegos, atendiendo a \cite{Hunicke04mda:a} son dinámicas, mecánicas y componentes.

\paragraph{\index{Dinámicas}}

Por ejemplo, se considerarían dinámicas las reglas, las emociones, la narrativa, la progresión y las relaciones personales.

\paragraph{\index{Mecánicas}} Las mecánicas son los procesos que 

Por ejemplo se considerarían mecánicas los retos, la suerte, la competición y la cooperación, el feedback, las recompensas, la adquisición de recursos, las transacciones y los estados de finalización.

\paragraph{\index{Componentes}} Los componentes son las instancias específicas de las mecánicas y las dinámicas. 

Serían componentes los logros, los avatares, los monstruos finales\footnote{Batalla muy difícil que tiene lugar al final de cada nivel.}, los combates, el desbloqueo de contenido, los niveles, las misiones, los equipos, las posesiones y el grafo social.
%
Además, 3 componentes que merecen una mención específica son los puntos (points), las medallas (bdages) y los rankings (leaderboards), comúnmente llamados \gls{PBL}.

\paragraph{La gamificación es más que \gls{PBL}:} Es habitual confundir gamificación con un sistema de recompensas basado en \gls{PBL}. 
%
La gamificación trata de crear una experiencia centrada en la persona. 
%
La implementación de un sistema de recompensas enfocado a realizar con eficiencia la función, dejando de lado la experiencia de los jugadores no puede ser considerada una gamificación y de hecho puede ser contraproducente.


\subparagraph*{Puntos}

\subparagraph*{Medallas}

\subparagraph*{Leaderboards}

Los leaderboards desmotivan al 80\% de la gente. 

Podríamos definir competición utilizando la definición de \cite{Crawford_CompetitionDef} Competition is when students are “constrained from impeding each other and instead devote the entirety of their attentions to optimizing their own performance.”

Crean una competición con la que hay que tener cuidado. Leyendo \cite{CompetitionInEd}...

\subsubsection{Impacto de la motivación}

\paragraph{Posibles peligros}
\label{PosiblesPeligros}

Ya hemos constatado un posible peligro al describir el componente \textit{leaderboard}. Además, hay otros.

Más competidores, menos motivación. \cite{n-effect}

Premios para los ganadores deberían ser de poca importancia o incluso simbólicos para asegurar que el esfuerzo de los estudiantes es intrínseco y no está dirigido por la expetativa del premio \cite{CompetitionInEd}.

\section{Educación gamificada}

Hasta ahora hemos hablado en términos generales de la gamificación. 
%
En esta sección trataremos de estudiar cómo se adapta esa teoría general al ecosistema de un aula.

\cite{lee2011gamification} constata varias cosas.

Hay otros estudios interesantes como \cite{Hanus2015152} y \cite{ReviewGamificationInEducation}.

\subsection{Potencial de la gamificación en el aula}

En la sección \ref{sec:EstadoEducacionMates} hemos constatado algunos problemas. 
%
Vamos a ir viendo cómo la Gamificación puede atajar esos problemas concretos y también, qué problemas se quedarían sin atajar.

\subsubsection{¿Es una metodología inclusiva?}

Esta pregunta es importante, pero no hay que descartar la metodología en caso de que la respuesta sea negativa, ya que también es importante plantearse:  ¿Es una metodología \textbf{más inclusiva} que la tradicional?
%
Tal vez no es una metodología perfecta, pero sí una metodología que aporta mejoras.

