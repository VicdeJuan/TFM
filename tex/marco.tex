
\section{La Gamificación}

\coment{Esto es lo que correspondería al marco teórico.}

En esta sección procedemos a describir a nivel general la Gamificación. 
%
La Gamificación entendida como una técnica que puede ser aplicada a muchos ámbitos, no sólo al educativo. 
%
Primeramente trataremos de clarificar el concepto de gamificación, porqué y cómo se puede gamificar un contexto.
%
Una vez detallado, en la siguiente sección trataremos de aplicar la gamificación al aula.



\coment{Qué es la Gamificación}

Gamificación o ludificación, optamos por Gamificación porque...

Atendiendo a \cite{GamificationDef} podemos definir \concept{Gamificación} como \textit{el uso de elementos de juegos en contextos no lúdicos}. 

Algunos de los elementos que hacen atractivos los juegos son
\textbf{Feedback instantáneo}
poder tomar \textbf{decisiones importantes},
\textbf{bucles de atracción y de fracaso},
\textbf{niveles de dificultad progresiva}... La idea es incorporar esos elementos en algún contexto no lúdico (en general, no sólo en la educación\footnote{No querría centrarme sólo en la educación para dar una visión más general de la gamificación como técnica en muchos campos: marketing, negocios... Creo que ofrece un punto de vista más riguroso y científico hablar de Gamificación en general y no sólo Gamificación en la educación.}).


\coment{Por qué gamificar}

Alguna lectura sobre porqué gamificar conceptualmente.

Porque funciona. Conclusiones de \cite{EmpiricalGamification}


\subsection{Diferencias entre Juegos serios y Gamificación}

Son conceptos distintos. Los \concept[Juegos serios]{juegos serios} son \cite{GamificationDef}... mientras que la gamificación consiste en...

Aunque los juegos serios pueden ser interesantes (conclusiones de \cite{MetaSerious}), se salen de este estudio.


\subsection{Elementos de la Gamificación}

\paragraph{Pensando como un diseñador}
\paragraph{La anatomía de la diversión}

Los elementos de la Gamificación son los elementos que podemos encontrar en los juegos, atendiendo a \cite{Hunicke04mda:a} son dinámicas, mecánicas y componentes.


\subsubsection{Dinámicas}

\subsubsection{Mecánicas}

\subsubsection{Componentes}



\paragraph{Puntos ...} \gls{pbl}

\subparagraph*{Puntos}

\subparagraph*{Medallas}

\subparagraph*{Leaderboards}

Los leaderboards desmotivan al 80\% de la gente. 

Podríamos definir competición utilizando la definición de \cite{Crawford_CompetitionDef} Competition is when students are “constrained from impeding each other and instead devote the entirety of their attentions to optimizing their own performance.”

Crean una competición con la que hay que tener cuidado. Leyendo \cite{CompetitionInEd}...

\subsubsection{Impacto de la motivación}

\paragraph{Posibles peligros}

Ya hemos constatado un posible peligro al describir el componente \textit{leaderboard}. Además, hay otros.

Más competidores, menos motivación. \cite{n-effect}

Premios para los ganadores deberían ser de poca importancia o incluso simbólicos para asegurar que el esfuerzo de los estudiantes es intrínseco y no está dirigido por la expetativa del premio \cite{CompetitionInEd}.

\section{Educación gamificada}

Hasta ahora hemos hablado en términos generales de la gamificación. 
%
En esta sección trataremos de estudiar cómo se adapta esa teoría general al ecosistema de un aula.

\cite{lee2011gamification} constata varias cosas.

Hay otros estudios interesantes como \cite{Hanus2015152} y \cite{ReviewGamificationInEducation}.

\subsection{Potencial de la gamificación en el aula}

En la sección \ref{sec:EstadoEducacionMates} hemos constatado algunos problemas. 
%
Vamos a ir viendo cómo la Gamificación puede atajar esos problemas concretos y también, qué problemas se quedarían sin atajar.

\subsubsection{¿Es una metodología inclusiva?}

Esta pregunta es importante, pero no hay que descartar la metodología en caso de que la respuesta sea negativa, ya que también es importante plantearse:  ¿Es una metodología \textbf{más inclusiva} que la tradicional?
%
Tal vez no es una metodología perfecta, pero sí una metodología que aporta mejoras.

