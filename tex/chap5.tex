%!TEX root = ../TFM.tex

\chapter{Conclusiones}
\label{chap:conclusiones}

Con esta propuesta metodológica se espera que los alumnos participen de un proceso de enseñanza y aprendizaje divertido, que rompa el círculo vicioso de la actitud de rechazo hacia las matemáticas, logrando una mayor motivación y gusto por la asignatura.
%
Se espera también que el aprendizaje de los alumnos sea funcional, significativo y duradero, ya que los contenidos matemáticos trabajados se utilizan como base para construir muchos otros conceptos más complicados, como las funciones, las matrices, etc.
Asimismo, es de esperar que, a la luz de los resultados que se obtengan, se pueda ampliar la propuesta a una \gls{PDA} para toda la asignatura, lo que permitiría un proceso de enseñanza aprendizaje más significativo a lo largo de todo el curso.

Será importante que tras la aplicación de esta propuesta se realice el trabajo de reflexión y autoevaluación para mejorar el diseño de la gamificación.
%
En el diseño de esta gamificación se han tomado decisiones (como la no inclusión de avatares) debido a que ha sido diseñada para una Unidad Didáctica.
%
Si se diseñara la gamificación para un curso escolar entero existen determinados elementos del juego que resultarían beneficiosos si se incluyeran.


Por otro lado, a la luz de los resultados que se obtengan, sería interesante diseñar la gamificación interdisciplinarmente.
%
Por ejemplo, en Geografía e Historia o incluso en Lengua y Literatura, que en este curso se estudia la literatura medieval, que incluye \comillas{El Cantar del Mío Cid} se podrían crear sinergias muy enriquecedoras para la experiencia escolar de los estudiantes.

Por último, con vistas a seguir mejorando la propuesta, sería interesante explorar otras metodologías como el aprendizaje cooperativo o los ejercicios de modelización para diseñar los trabajos cooperativos e individuales de una manera óptima, aprovechando todos el conocimiento disponibles en la investigación actualmente.

