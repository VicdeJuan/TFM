%!TEX root = ../TFM.tex
\chapter{Documentos y ejercicios para proponer en el aula}


\section{Expresiones algebraicas y su valor numérico}
\label{app:DocModel}



\section{Test de la sesión 1}
\label{test:ses1}

Test en edmodo sobre la suma y resta de polinomios, con uno de valor numérico\footnote{Considera p,q,s=p+q. Calcula p(2),q(2). Sin calcular s(2), ¿cuánto debería ser? Compruébalo.}
%
Estudia si existe alguna relación entre esos tres valores 

\section{Acertijo Cooperativo (sesión 2)}
\label{app:ses2:coop}


Mapa: Puntos en un mapa con etiqueta, una leyenda que asigne a cada etiqueta un polinomio y pasos para conseguir.
%
Trabaja la multiplicación


\section{Deberes sesión 2}
\label{app:ses2:deberes}



\section{Acertijo Cooperativo (sesión 3)}
\label{app:ses3:coop}


Mapa: Puntos en un mapa con etiqueta, una leyenda que asigne a cada etiqueta un polinomio y pasos para conseguir.
%
Trabaja la división.

\section{Documentos para la sesión 4}

\subsection{Reto individual}
\label{app:ses4:indiv}

Reto de unir polinomios con su factorización.

\subsection{Deberes}
\label{ses4:deberes}

$x^4-16 = (x^2+4)(x^2-4) = (x^2+4)(x+2)(x-2)$

\section{Reto cooperativo (sesión 6)}
\label{ses6:coop}

\subsection{Deberes}
\label{ses6:deberes}



\section{Reto final (sesiones 7 y 8)}
\label{ses7:indiv}

\subsection{Alumno 1}

%!TEX root = ../TFM.tex

\subsection{Alumno 2}
%!TEX root = ../TFM.tex


\subsection{Alumno 3}
%!TEX root = ../TFM.tex


\subsection{Alumno 4}
%!TEX root = ../TFM.tex



\subsection{Parte cooperativa}
\label{ses7:coop}


\section{Examen}
\label{examen}

\section{Autoevaluación}
\label{app:autoeval}