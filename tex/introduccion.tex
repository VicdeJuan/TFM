\label{sec:intro} % etiqueta para poder referenciar luego en el texto con ~\ref{sec:intro}
\pagenumbering{arabic} % para empezar la numeración de página con números

En este capítulo se introduce el proyeto. 
Debería tener información general sobre  el mismo, dando la información sobre el contexto en el que se ha desarrollado.

%No te olvides de echarle un ojo a la página con los cinco errores de escritura más frecuentes\footnote{\url{http://www.tallerdeescritores.com/errores-de-escritura-frecuentes}}.

\section{Ideas interesantes}

\begin{itemize}
	\item \url{https://openbadges.org/developers/} para clase de informática. 1 punto por badget conseguido. Hay 15 y tu eliges cuales quieres.
\end{itemize}

\section{Estructura de la memoria}
\label{sec:estructura}

En esta sección se debería introducir la esctura de la memoria. Así:

\begin{itemize}
  \item En el primer capítulo se hace una intro al proyecto.
  
  \item En el capítulo~\ref{chap:objetivos} se muestran los objetivos del proyecto.
  
  \item A continuación se presenta el estado del arte.
  
  \item \ldots
\end{itemize}




