%!TEX root = ../TFM.tex
%\addcontentsline{toc}{chapter}{Resumen} % si queremos que aparezca en el ?dice
\markboth{RESUMEN}{RESUMEN} % encabezado

Within the Spanish education environment it is necessary to improve its quality and to innovate the methodologies it use.
%
The last education law approved, Ley Orgánica para la Mejora de la Calidad Educativa, has that claim included in its name.

Eventhough there are not enough, there are increasingly more teachers and institutions including the new advances to their praxis inside the classrooms.
%
It is important to support these initiatives and provide them with resources and possibilities.  
%
This paper tries precisely to do so, offer a detailed study of Gamification as a strategy for the classroom followed by a concrete proposal, designed and detailed, ready to be implemented inside the Matemáticas Orientadas a las Ciencias Académicas course in 3º ESO.


\begin{keywordsEn}
Gamification, Players Taxonomy, Motivation and Flow theories, PISA Reports , Didactic Unit, Mathematics, High School, LOMCE
\end{keywordsEn}
