%!TEX root = ../TFM.tex

\newcommand{\arab}{al-Karaji}
\newcommand{\Arab}{Al-Karaji}
\newcommand{\logro}[2]{\labeltext{#1\xspace}{logro::#2} #1\xspace}


\chapter{Gamificar las Matemáticas}

%Lo que debes reflfejar en este capítulo es cómo lo has hecho, cual es tu objetivo, como organizas el aula, como lo evaluas y qué resultados obtienes.

Una vez explicitados los fundamentos sobre los que basamos nuestra propuesta metodológica, procedemos a su descripción.


\section{Diseño de la gamificación}

En el apartado \ref{PasosGamificar} se ha tratado, en términos generales, la propuesta para diseñar una gamificación de \cite{kapp2013gamification}. 
%
Se va a utilizar esa propuesta para guiar el proceso de diseño de la gamificación. 
%
Para ello, se irán recorriendo las etapas propuestas.

El primer paso consiste en estudiar el problema que se quiere resolver. 
%
Éste ha sido tratado con amplitud en la sección \ref{chap:intro}, estudiando los comportamientos existentes. 
%
Por otro lado, el comportamiento deseado sería que los estudiantes estuvieran más motivados y comprometidos con la asignatura, dispuestos a trabajar activamente en clase y trabajaran en casa.
%
Además, el objetivo competencial está definido por el \bocm y se detallará más adelante (ver \ref{tbl:Matrizdetodo}).
%
El último aspecto de las preguntas base es la valoración de los aspectos que puedan darnos retroalimentación sobre el proceso.
%
La manera en la que se va a conseguir eso es mediante el diseño de un itinerario de logros y puntos que los alumnos van recorriendo.
%
De esta manera se puede anticipar fácilmente si hay algún alumno rezagado simplemente observando los logros adquiridos y los puntos.
%
No obstante, en cada sesión y tarea se especificará un mínimo a alcanzar por parte de los alumnos.

En el segundo paso se trabajan las preguntas prácticas.
%
El público objetivo de la gamificación son alumnos de secundaria del sistema educativo español. 
%
Debido a que esta gamificación no se plantea para ser aplicada en un centro en concreto, no se pueden determinar con exactitud los conocimientos previos y habilidades ni los tipos de jugadores concretos a los que esta propuesta se dirige.
%
No obstante, en la sección \ref{sec:taxonomy} se han tratado los tipos de jugadores para que el docente, en el momento de implantación de esta gamificación pueda definir los tipos de jugadores concretos a quienes se dirige la gamificación.
%
Por otro lado, la definición logística y técnica de la gamificación se encuentra especificado a lo largo de la descripción de la unidad didáctica (ver sección \ref{sec:UD}), teniendo en cuenta los recursos existentes y las habilidades del docente.

Asimismo, el sistema de valoración también está incluido en la unidad didáctica, anticipando durante el diseño los posibles fallos del sistema para conseguir una evaluación lo más justa posible.

El último paso de la propuesta de \cite{kapp2013gamification} sería la puesta en marcha del sistema. 
%
Desafortunadamente, esta gamificación no ha sido llevada al aula todavía.

\subsection{Marco legal de aplicación}

La \lomce establece que: "Necesitamos propiciar las condiciones que permitan el oportuno cambio metodológico, de forma que el alumnado sea un elemento activo en el proceso de aprendizaje" 
%
En este sentido, nuestra propuesta metodológica se apoya en las indicaciones establecidas por la \lomce para el ejercicio de la labor docente.

La \lomce establece condiciones generales que son concretadas por el \boe, estableciendo los contenidos que se deben tratar en cada curso.
%
Después, corresponde a cada Administración la elaboración del currículo básico para su comunidad autónoma.
%
En este caso, recurrimos al \bocm.


\subsection{Elección del curso para Gamificar}

El curso elegido para desarrollar la propuesta es Matemáticas Orientadas a las Ciencias Académicas de 3 de ESO por ser 3 de ESO el curso en el que se constata que se ha producido un aumento del rechazo hacia las Matemáticas (ver sección \ref{sec:estudioNacional}) 
%
Además, es también el curso en el que la percepción de la asignatura como divertida menor en este curso (40,46\% en 3 de ESO frente a 54,46\% en Bachillerato y 68,51\% en 1 de ESO) \cite{ActitudesHaciaMates}.

Dentro del curso se ha elegido desarrollar una unidad didáctica sobre los polinomios porque es el primer curso en el que se tratan los polinomios y su mal aprendizaje puede lastrar al alumno hasta los primeros cursos universitarios.
%
La factorización de polinomios, por ejemplo, es un recurso importante en la diagonalización de matrices, contenido fundamental en Álgebra de cualquier grado en Ingeniería o Administración y Dirección de Empresas.
%
Si en este curso se consigue un aprendizaje consolidado y duradero, los estudiantes podrán desempeñar y aprovechar mejor los cursos venideros.


\section{Unidad didáctica gamificada}

\label{sec:UD}
%
Esta unidad didáctica pertenece a la asignatura Matemáticas Orientadas a las Ciencias Académicas de 3 ESO. 
%
Los contenidos de la asignatura, definidos en el \boe y matizados por el \bocm, se dividen en 5 bloques:
1) Procesos métodos y actitudes en las matemáticas. 
2) Números y álgebra: Números racionales, números decimales, polinomios, sucesiones, ecuaciones y sistemas.
3) Geometría: geometría plana, geometría del espacio, globo terráqueo, uso de herramientas tecnológicas.
4) Funciones.
5) Estadística y probabilidad.

Esta unidad didáctica pertenece al bloque de Números y Álgebra para tratar el punto 6 de los contenidos del bloque 2 según el \bocm: \comillas{Polinomios. Expresiones algebraicas.}
%
Este punto incluye:
%
\comillas{Transformación de expresiones algebraicas, igualdades notables, operaciones elementales con polinomios. 
%
Ecuaciones de primer y segundo grado con una incógnita y resolución por el método algebraico y gráfico de ecuaciones de primer y segundo grado.}
%
Más adelante, en la tabla \ref{tbl:Matrizdetodo} se expondrán más claramente los Contenidos, Criterios de Evaluación y \eaes pertinentes.

Esta unidad didáctica desarrollará todo lo relativo a polinomios y su manejo, excluyendo las ecuaciones y sistemas de ecuaciones que corresponderán a otra unidad didáctica diferente.

\subsection{Conocimientos previos}

Es fundamental que los alumnos recuerden las propiedades de las potencias de exponentes naturales con las que llevan trabajando al menos 2 cursos.
%
Asimismo, sería conveniente que los estudiantes recordaran qué es una expresión algebraica y cómo calcular su valor numérico.
%
No obstante, esto último será repasado en el desarrollo de esta unidad didáctica para corregir posibles desequilibrios entre los niveles de los alumnos.



\subsection{Contenidos y Estándares de Aprendizaje Evaluables}

En la tabla \ref{tbl:Matrizdetodo} se encuentran resumidos los Contenidos, Criterios de Evaluación y  Estándares de Aprendizaje Evaluables de la unidad didáctica desarrollada.

\begin{table}[hbt]
\centering
\caption{Matriz de Contenidos, Criterios de Evaluación y  Estándares de Aprendizaje Evaluables de la unidad didáctica}
\label{tbl:Matrizdetodo}
\begin{tabular}{|p{0.24\linewidth}|p{0.3\linewidth}|p{0.42\linewidth}|}
\hline
 \multicolumn{1}{|c|}{Contenidos} & \multicolumn{1}{|c|}{Criterios de Evaluación} & \multicolumn{1}{c|}{\eaes}
\\\hline

\mylabel{C261}{Cont. 2.6.1} Transformación de expresiones algebraicas. 
&
\multirow{3}{\linewidth}{\mylabel{CE23}{C.E. 2.3} Utilizar el lenguaje algebraico para expresar una propiedad o relación dada mediante un enunciado, extrayendo la información relevante y transformándola.\vfill}
& 
\mylabel{EAE3.1}{E.A.E. 3.1}: Realiza operaciones con polinomios y los utiliza en ejemplos de la vida cotidiana.
\\\cline{1-1} \cline{3-3} 

\mylabel{C262}{Cont. 2.6.2} Igualdades notables. 
&
& 
\mylabel{EAE3.2}{E.A.E. 3.2}: Conoce y utiliza las identidades notables correspondientes al cuadrado de un binomio y una suma por diferencia, y las aplica en un contexto adecuado. 
\\\cline{1-1} \cline{3-3} 

\mylabel{C263}{Cont. 2.6.3} Operaciones elementales con polinomios. 
&
&
\mylabel{EAE3.3}{E.A.E. 3.3}: Factoriza polinomios de grado 4 con raíces enteras mediante el uso combinado de la regla de Ruffini, identidades notables y extracción del factor común.
\\\hline
\end{tabular}
\end{table}
\FloatBarrier

\subsection{Metodología:}

La metodología principal aplicada en esta unidad didáctica será la gamificación.
%
Ésta se apoyará de otras estrategias metodológicas como el trabajo cooperativo para algunos trabajos concretos, el trabajo autónomo y la explicación en la pizarra o con proyecciones por parte del docente.

Con esta metodología específica se pretende aportar las condiciones necesarias para que el aprendizaje de los alumnos sea significativo, funcional y duradero.
%
También aporta un enfoque lúdico y emocionante al aprendizaje y es bien sabido que sin emoción no se produce aprendizaje.
%
Además, esta elección metodológica permite hacer protagonista al alumno durante el proceso logrando que se involucre activamente en el proceso de enseñanza-aprendizaje.
%
Por último es una metodología tremendamente flexible que permite sinergias con otras metodologías que serán aprovechadas, como el aprendizaje cooperativo ya mencionado y la clase invertida.


Otro aspecto de la \todo{Tal vez debería ir en recursos...} estrategia metodológica es la utilización de herramientas \gls{TIC}, sobre las que la \lomce establece que \comillas{serán una pieza fundamental para producir el cambio metodológico que lleve
a conseguir el objetivo de mejora de la calidad educativa.}
%
Los ejercicios de trabajo autónomo por parte del estudiante en horario no lectivo serán realizados en una plataforma online (\textit{edmodo}) de tal manera que el estudiante pueda recibir feedback inmediato de su desempeño.
%
Salvo que se especifique lo contrario, no es relevante el tiempo empleado en la resolución de estos ejercicios sino su resolución satisfactoria o insatisfactoria.
%
Se fomentará la corrección en las respuestas más que la rapidez con un bonus acumulativo por respuestas correctas, además de ofrecer la posibilidad de retomar el ejercicio un número de veces determinado, para facilitar el aprendizaje a partir de los errores y los fallos.
%
Estos parámetros dependerán de cada set de ejercicios y se especificarán pertinentemente.
%
En otros casos, se utilizará la clase invertida, proponiendo un vídeo para trabajar en casa y trabajar en clase los contenidos aprendidos con el vídeo.


En esta unidad didáctica se está priorizando el aprendizaje activo de los contenidos y competencias otorgando menos importancia a otras habilidades transversales y necesarias como tomar apuntes y notas en clase. 
%
Esta competencia es muy útil (aunque cada vez más en desuso) y se priorizará en otras unidades didácticas del curso. 
%
Se ofrecerá a los alumnos un libro de texto de apoyo para que puedan consultar cuando necesiten cualquier aspecto teórico.
%
Para esta unidad didáctica, al no estar suscrita a ningún centro utilizaremos un material didáctico elaborado por \citeauthor{MareaVerde} en \citeyear{MareaVerde} en el movimiento de la Marea Verde \citep{MareaVerde}.


\subsubsection{Concreciones metodológicas}

La narrativa de la Gamificación consistirá en la simulación de un instituto de investigación histórica.
%
Los alumnos son los historiadores con una mayor competencia matemática de Europa y se necesita su ayuda para encontrar un tesoro árabe que se escondió en Madrid en 1083, justo antes de la reconquista cristiana.

Se han encontrado una serie de documentos que no se ha conseguido descifrar en su totalidad y se cree que los alumnos van a ser capaces de resolverlo.
%
%
Este documento utiliza las matemáticas de Abu Bekr ibn Muhammad ibn al-Husayn al-Karaji, el primero en definir los monomios $x$,$x^2$... y proporcionar reglas para su producto, es decir, el primero en definir las bases del álgebra\citep{MatArabe}.
%
Para poder resolver el enigma será necesario que aprendamos las matemáticas que se utilizaban en Al-Ándalus.

Será necesario que en su cuaderno incorporen un diario del investigador, en el que ir tomando notas de lo que les resulte importante, dejando constancia de los avances que van realizando, etc.
%
En cada clase se irán obteniendo partes de las coordenadas geográficas de un lugar de Madrid y se irán aprendiendo herramientas para resolver el problema.

%%%%%%%%%%%%% \todo{Definir el lugar donde está el tesoro}

%%%%%%%%%%%%%%%%%%%%%%%%%%%%%%%%%%%%%%%%%%%%%%%%%%%%%%%%%%%%%%%%%%%%%%%%%%
%%%%%%%%%%%%%%%%%%%%%%%%%%%%%%%%%%%%%%%%%%%%%%%%%%%%%%%%%%%%%%%%%%%%%%%%%%
%%%%%%%%%%%%%%%%%%%%%%%%%%%%%%%%%%%%%%%%%%%%%%%%%%%%%%%%%%%%%%%%%%%%%%%%%%
%%%%%%%%%%%%%%%            TEMPORALIZACIÓN           %%%%%%%%%%%%%%%%%%%%%
%%%%%%%%%%%%%%%%%%%%%%%%%%%%%%%%%%%%%%%%%%%%%%%%%%%%%%%%%%%%%%%%%%%%%%%%%%
%%%%%%%%%%%%%%%%%%%%%%%%%%%%%%%%%%%%%%%%%%%%%%%%%%%%%%%%%%%%%%%%%%%%%%%%%%
%%%%%%%%%%%%%%%%%%%%%%%%%%%%%%%%%%%%%%%%%%%%%%%%%%%%%%%%%%%%%%%%%%%%%%%%%%
%%%%%%%%%%%%%%%%%%%%%%%%%%%%%%%%%%%%%%%%%%%%%%%%%%%%%%%%%%%%%%%%%%%%%%%%%%


\subsection{Temporalización}

La unidad didáctica se divide en 10 sesiones.

\subsubsection{Sesión 0: Introducción}


\paragraph{Contenidos}

Los contenidos a tratar en esta sección serán:
\begin{itemize}

\item Expresiones algebraicas.
\item Monomios y operaciones básicas.
\item Valor numérico.
\end{itemize}

\paragraph{Desarrollo de la sesión: }

En esta primera sesión introductoria se explicará la narrativa de la gamificación: 
%
los alumnos son investigadores que buscan desentrañar un misterio y para ello, a veces formarán equipos de investigación, otras veces trabajarán autónomamente.
%
El objetivo no será ganar y ser el primero en descubrirlo, sino que, como todos somos investigadores al servicio de la ciencia, queremos desentrañar la verdad y que todos la entendamos.
%
Se trata de colaborar.

El misterio a resolver tiene ya 10 siglos.
%
\Arab, nacido en el actual Irak, dejó un mapa y unos acertijos para desenterrar un tesoro que se dejó en Madrid.
%
Consideró que sólo un árabe que supiera de matemáticas tanto como él sería capaz de resolverlo.
%
Han pasado varios siglos y se cree que con los conocimientos actuales será posible resolverlo.
%
El descubrimiento del mapa y los acertijos es muy reciente, por eso todavía no se ha desenterrado el tesoro.

Se explicará también la gamificación y que para la evaluación de su labor como investigadores se utilizarán logros y medallas que podrán ir adquiriendo a medida que avancen la investigación y puntos de reputación, según las tareas que realicen.

Lo primero es necesario asegurar que los alumnos disponen de la competencia matemática suficiente para llevar a cabo la tarea encomendada.
%
Para ello se realizará en esta primera sesión una prueba de aptitud para la tarea.
%
Esta prueba de nivel competencial se realizará en el aula de informática con \textit{Kahoot}.
%
Servirá como repaso de los contenidos de otros años: expresiones algebraicas, operaciones con monomios, valor numérico.
%
Cada bloque de preguntas irá precedido de un pequeño vídeo que refresque los contenidos.
%
A los puntos otorgados por \textit{Kahoot} se le incorporará un bonus polinómico por preguntas acertadas para priorizar la precisión frente a la rapidez.
%
Los alumnos que la hayan superado obtendrán el logro de \logro{Investigador Apto}{investigador_apto}.
%
Los alumnos que no lo hayan superado en clase, tendrán la opción de repetir la prueba en casa para conseguir el logro de \ref{logro::investigador_apto}.



\subsubsection{Sesión 1: Introducción a los polinomios}

\paragraph{Contenidos:}
Los contenidos a tratar en esta sesión son:
\begin{itemize}
\item Qué es un polinomio.
\item Grado de un polinomio.
\item Suma y diferencia de polinomios.
\item Producto de un monomio por un polinomio.
\item Factor común de monomios.
\end{itemize}


\paragraph{Desarrollo de la sesión} Trabajo individual sobre una práctica de modelización.


Se les entrega un supuesto \comillas{primer documento de \arab} (expuesto en \ref{app:DocModel}) en el que él mismo explica algunos conceptos.
%
Esta es una segunda prueba para comprobar que están a la altura del reto.
%
Este documento ya ha sido trabajado y resuelto, pero les puede servir para practicar antes de enfrentarse a los, todavía, irresolutos.
%
Durante la resolución del acertijo el docente paseará entre los alumnos resolviendo dudas y ayudando a quienes les esté costando más. 
%
Si resuelven con éxito este trabajo autónomo obtendrán el logro \logro{Investigador Nato}{investigador_nato}.
%
Si no lo han podido resolver en clase, será necesaria su resolución para la obtención del logro.

En caso de que el trabajo en clase se haya desarrollado satisfactoriamente y todos los alumnos hayan conseguido descifrar el acertijo, se propondrá un pequeño test online (\ref{test:ses1}) para seguir practicando y ejercitándose en el álgebra.
%
Excepcionalmente, en el caso de que el trabajo haya resultado muy complicado a los alumnos se podría emplear una sesión más en realizar los ejercicios del test en clase, con las explicaciones pertinentes del profesor.

Para la resolución de los ejercicios en casa dispondrán de la teoría y ejemplos resueltos en la ficha de trabajo del día además del libro de texto de apoyo ya mencionado.

Además, en esta sesión se pondrá en marcha el logro \logro{Contextualizador}{contextualizador}, que los alumnos podrán obtener por la aplicación de alguno de los contenidos de la clase de matemáticas en su vida cotidiana y su posterior explicación a la clase.


\subsubsection{Sesión 2: Operaciones con polinomios I}

\paragraph{Contenidos}
\begin{itemize}
	\item Multiplicación de polinomios: propiedad asociativa y distributiva.
	\item Factor común de polinomios.
\end{itemize}

\paragraph{Desarrollo de la sesión}

Se les presentará el primer acertijo real irresoluto de la investigación.
%
Debido a la complejidad se trabajará cooperativamente por grupos de 4 decididos por el profesor atendiendo a la teoría del aprendizaje cooperativo. 
\todo{cita}

Se les explicará a los alumnos que en cualquier trabajo siempre hay una fase previa de formación y para que puedan enfrentarse con alguna garantía al acertijo es necesario que conozcan algunas herramientas matemáticas, como la multiplicación de polinomios.
%
Se procederá a una explicación breve por parte del profesor en la pizarra sobre la multiplicación de polinomios explicando la propiedad asociativa y distributiva, incluyendo algunos ejemplos.

Una vez finalizada la explicación, se les entregará el acertijo expuesto en el anexo \ref{app:ses2:coop}
%
La resolución del acertijo es un número entero que corresponderá a la parte entera de la latitud del lugar en el que se escondió el tesoro.



Se les propondrá que realicen trabajo en casa para ejercitarse y poder resolver con garantías los siguientes retos.
%
Se les anunciará que para el próximo día se trabajará sobre un documento no resuelto a día de hoy y será necesario que dominen las matemáticas básicas que han aprendido en esta sesión.
%
Se les propondrán varios ejercicios en la plataforma online de los que tendrán que seleccionar un número determinado. 
%
En principio serían 3, pero este valor puede ser modificado dependiendo del desempeño de los alumnos durante la sesión.
%
Cada ejercicio tendrá asociado una puntuación: los ejercicios más difíciles otorgan más puntos, los ejercicios más fáciles, menos puntos.
%
Se especificará un número mínimo de puntos a conseguir para que los alumnos practiquen en casa. 
%
Para motivar que realicen los ejercicios, introducimos el logro \logro{Investigador Comprometido}{investigador_comprometido} nivel 1 que se obtiene si se obtiene el número de puntos mínimo\footnote{Los puntos correspondientes a la realización de los 3 ejercicios más fáciles}.
%
Podrán acceder al nivel 2 de este logro si realizan todos los ejercicios propuestos, aunque sólo recibirán puntos de 3 ejercicios.
%
Los ejercicios propuestos se encuentran en \ref{app:ses2:deberes}

\Justificacion{de la posibilidad de elección}
	%
	es bien sabido que los deberes no resultan motivadores para los alumnos. 
	%
	Para fomentar su realización se están utilizando 2 técnicas: 
	%
	la primera consiste en la explicación sobre la utilidad que puede tener para ellos (resolver mejor los próximos enigmas) y busca despertar una motivación intrínseca de competencia a corto plazo y no a largo plazo como tradicionalmente se hace (para hacer mejor el examen, para ser buen estudiante y tener un futuro asegurado, etc.).
	%
	Por otro lado, buscando fomentar su sentido de autonomía se les ofrece la posibilidad de elegir sobre los deberes a realizar. 
	%
	Pueden hacer un número variable de ejercicios dependiendo de su interés, habilidad y motivación.
	%
	Es importante la libertad de elección y que los jugadores sientan que esas elecciones son importantes \citep{werbach2012win}.

	Además, al ser resueltos en una plataforma online, son corregidos en el momento, provocando una retroalimentación sobre su desempeño inmediata
	%
	\footnote{Una posible mecánica de la gamificación según \citeauthor{werbach2012win}, descrita en \ref{mecanicas}.}.
	%
	Si los ejercicios propuestos suponen un desafío al alumno, se podría inducir un estado de flow llevando al alumno a querer resolver todos los ejercicios o por lo menos, a disfrutar durante su realización.

\subsubsection{Sesión 3: Operaciones con polinomios II}

\paragraph{Contenidos}
\begin{itemize}
	\item División de polinomios.
	\item Algoritmo de la división
\end{itemize}

\paragraph{Desarrollo de la sesión}

El desarrollo de esta sesión será el mismo que el de la sesión 2.
%
Una primera parte de exposición sobre el contenido necesario para la resolución del enigma y una ficha de trabajo cooperativo para su resolución que se encuentra en \ref{app:ses3:coop}.
%
El correcto resultado del acertijo les conducirá a otro número entero, que corresponderá a la parte entera de la longitud de la localización del tesoro.

Debido a que los alumnos desconocen que los 2 números enteros corresponden a la latitud y a la longitud, mediante un diálogo se les conducirá a esa conclusión.
%
Los conceptos de latitud y longitud se trabajan por primera vez en Ciencias Sociales de 1 de la ESO, según \bocm.


El trabajo a realizar en casa será encontrar la localización en un mapa de Madrid del recuadro que corresponde a la parte entera de la longitud y latitud.
%
Tendrán que saber qué número de los obtenidos corresponde a latitud y cuál a longitud.
%
Que impriman el mapa para pegarlo en su cuaderno de investigador.

\Justificacion{}
%
La realización de este ejercicio puede parecer que no trabaja la competencia matemática. 
%
Pero para la obtención del mapa es necesario tener claros los conceptos de parte entera y parte decimal.
%
Si la parte entera de la longitud es 3, la longitud real podría ser desde 3,01 hasta 3,99, lo que marca una gran diferencia.
%
Además, este ejercicio puede aumentar la involucración de los estudiantes en la narrativa de la gamificación y en el proceso.
%
Asimismo, supone un trabajo de la competencia digital (para encontrar el recuadro del mapa e imprimirlo).
%
Por otro lado, la división de polinomios se trabajará más adelante durante la unidad lo que permite aplazar la asimilación profunda del contenido.
%
Esta situación es diferente a la situación de la multiplicación de polinomios, ya que es una operación básica que deben interiorizar rápidamente ya que se utiliza para la división, para su posterior comprobación y para la factorización.
%
Por ello en la sesión de multiplicación es necesario trabajo en casa que trabaje el contenido, mientras que en la sesión de división no es imprescindible.



\subsubsection{Sesión 4: Identidades notables}

\paragraph{Contenidos}
\begin{itemize}
	\item Identidades notables: 
	\subitem Cuadrado de un binomio: $(x\pm a)^2 = \;x^2\pm 2xa + a^2$.
	\subitem Diferencia de cuadrados: $(x-a)(x+a) = x^2-a^2$.
\end{itemize}

\paragraph{Desarrollo de la sesión}

A quienes tengan el mapa en su cuaderno de investigador obtendrán un nivel más de su logro \ref{logro::investigador_comprometido}.

\Justificacion{} este ascenso inesperado es utilizado para transmitir el mensaje de que no todos los logros están sujetos a contingencias esperables que los alumnos conozcan.
%
En una buena gamificación también deben existir logros no sujetos a contingencias esperados. \citep{werbach2012win}.

Se propone para esta sesión que el docente diga que va a traer a un invitado especial a la clase de hoy.
%
Se dirá a los alumnos que el invitado que va a venir es un árabe de la escuela matemática de \arab y viene para enseñarles algunos contenidos específicos necesarios para la investigación y que él conoce a la perfección.
%
En realidad no habrá ningún invitado y se propone que simplemente sea el mismo docente disfrazado.
%
Esto será algo nuevo y muy llamativo que provocará una gran fijación en la memoria.
%
Podríamos conseguir lo que se denomina un \concept[Recuerdo Relámpago]{recuerdo relámpago}.
%f
Si todo sale según lo previsto, los alumnos recordarán toda su vida de la clase de matemáticas que les dio el docente disfrazado de árabe explicando las identidades notables, 
%
de una manera similar, en menor medida, a la que se grabó en la mente de muchas personas cuál era la actividad que estaban desarrollando con exactitud el día 11 de Septiembre de 2001 \citep{11s}.

Se empezará la sesión con un reto individual con el que podrán obtener puntos de reputación.
%
El docente tendría tiempo para disfrazarse mientras los alumnos se enfrentan al reto.
%
El reto consistirá en unir polinomios con su factorización (aunque todavía no sepan lo que es la factorización): lo harán multiplicando los paréntesis y uniendo con el polinomio correspondiente.
%
Aparecerán las 3 identidades notables varias veces para ir cogiendo soltura.
%
El reto se encuentra definido en el anexo \ref{app:ses4:indiv}.

Una vez finalizado el reto y otorgados los puntos, se procederá a una explicación en la pizarra sobre las identidades notables y su ámbito aplicación en todos los modos posibles:
%
para resolver multiplicaciones rápidamente, para resolver divisiones y para resolver ecuaciones.
%
Éste último punto servirá como introducción a la factorización.


Se propondrán ejercicios de deberes en \textit{edmodo} (ver \ref{ses4:deberes}) con el esquema utilizado anteriormente.


\subsubsection{Sesión 5: Factorización}

\paragraph{Contenidos}
\begin{itemize}
	\item Factorización. Conceptos teóricos, diferencia entre raíz y factor.
	\item Factorización utilizando: identidades notables, fórmula de resolución de ecuaciones de segundo grado, sacando factor común y aplicando los casos anteriores en polinomios de la forma: $P(x) = ax^3+bx^2+cx$.
\end{itemize}

\paragraph{Desarrollo de la sesión}

Se pone en marcha la medalla de \logro{Ojo Avizor}{avizor}.
%
Esta medalla tiene 3 subniveles y como máximo se puede ascender un nivel en cada clase.
%
Cada vez que en cualquier clase aparezca una identidad notable, cualquier alumno podrá levantar la mano para identificarla y resolverla. 
%
Si la resuelve correctamente obtendrá un subnivel del logro.

La sesión se desarrollará utilizando la exposición por parte del docente con abundantes ejemplos y ejercicios para practicar por parte de los alumnos.
%
La sesión comenzará enlazando con el final de la sesión anterior: utilización de identidades notables para encontrar los ceros un polinomio.
%
Se explicará que a estos ceros se les llama raíces y que con ellos se pueden construir factores para escribir el polinomio como un producto de términos de la forma $(x-a)$ con $a\in\real$.
%
A estos factores que se multiplican, se les llama factores y al proceso de escribir un polinomio como producto de sus factores se le denomina factorización.

A continuación se empleará todo el tiempo restante en factorizar polinomios.
%
El docente realizará primero uno utilizando un método (identidades notables) y se propondrá a los alumnos 2 polinomios para factorizar así.
%
Se realizará lo mismo con los otros 2 métodos: resolución de ecuaciones de segundo grado y factor común.


\subsubsection{Sesión 6: Ruffini y atajos}

\paragraph{Contenidos}
\begin{itemize}
	\item Algoritmo de factorización de Ruffini.
\end{itemize}

\paragraph{Desarrollo de la sesión}

Se les propondrá un reto cooperativo durante la primera media hora (ver \ref{ses6:coop}).
%
Este reto es muy completo y no lo conseguirán terminar porque incluirá una factorización de un polinomio de grado 4 y otro de grado 3 con término independiente.
%
Así, será necesario que escuchen e interioricen el método de Ruffini para factorizar polinomios.
%
Podrán practicar con los 2 polinomios del reto y con otros ejercicios propuestos por el docente con los que los alumnos podrán obtener puntos.
%
La resolución del reto correctamente les otorgará la parte decimal de la longitud de la localización del tesoro.

Se propondrán ejercicios de deberes en \textit{edmodo} (ver \ref{ses6:deberes}) con el esquema utilizado anteriormente.
%
En estos deberes se incorporará un vídeo con la explicación teórica sobre la factorización de polinomios con coeficiente principal distinto de 1.

\subsubsection{Sesiones 7 y 8: Reto de todo combinado}

\paragraph{Contenidos}
\begin{itemize}
	\item Repaso de todo, incluyendo la factorización de polinomios con coeficiente principal distinto de 1.
\end{itemize}

Se les entregará el último papiro sin resolver de la investigación para que lo resuelvan.
%
Es un reto complicado que implica aplicar con soltura todos los conocimientos aprendidos además de seguir aplicando conocimiento de años anteriores.


Se dispone de 2 sesiones para su realización: la primera se trabajará individualmente (cada miembro del equipo tendrá una ficha ligeramente diferente, ver \ref{ses7:indiv}) y la segunda se trabajará cooperativamente (ver \ref{ses7:coop}).
%
Para la segunda sesión será necesario que los alumnos hayan finalizado con éxito sus partes individuales, ya que el equipo necesitará los resultados obtenidos en la parte individual.

La correcta resolución del ejercicio cooperativo les dará la parte decimal de la latitud del lugar del tesoro.
%
En caso de que no les diera tiempo a resolverlo en clase, tendrían que terminarlo por su cuenta.

De cara a la preparación la prueba objetiva tendrán en \textit{edmodo} un quizz que podrán hacer a modo de auto-evaluación cuantas veces quieran.

\subsubsection{Sesión 9}

En esta sesión, no necesariamente la sesión siguiente a la sesión 8, tendrá lugar la prueba objetiva de la unidad (si se considera que haya una prueba objetiva para esta unidad, ver \ref{eval}).
%
Los ejercicios propuestos para el la prueba objetiva se encuentran en \ref{examen}.

\subsection{Recursos}

\subsubsection{Recopilación de los elementos de la gamificación }

Los recursos de la gamificación utilizados se encuentran descritos en la tabla \ref{Gamify:resumen}.

\begin{table}
\centering
\caption{Recopilación de los elementos de la gamificación utilizados}
\label{Gamify:resumen}
\begin{tabular}{|m{0.06\linewidth}|m{0.35\linewidth}|m{0.42\linewidth}|}
\hline
ID & Nombre & Descripción\\\hline
EG1\labeltext{EG1}{puntos} & Puntos de reputación & Todos los puntos que se obtienen son de reputación.\\\hline
EG2\labeltext{EG2}{apto} & Logro \ref{logro::investigador_apto} &\\\hline
EG3\labeltext{EG3}{nato} & Logro \ref{logro::investigador_nato} &\\\hline
EG4\labeltext{EG4}{comp} & Logro \ref{logro::investigador_comprometido} & Niveles.\\\hline
EG5\labeltext{EG5}{avizor} & Logro \ref{logro::avizor} & Niveles.\\\hline
EG6\labeltext{EG6}{context} & Logro \ref{logro::contextualizador} & Niveles.\\\hline
EG7\labeltext{EG7}{narrativa} & Narrativa & Los alumnos son investigadores de unos manuscritos árabes.\\\hline
EG8\labeltext{EG8}{explorador} & Medalla \logro{Explorador}{explorador} & Obtenible por hacerse la foto en el lugar del tesoro.\\\hline
EG9\labeltext{EG9}{eleccion} & Elección & Posibilidad de elección de los deberes a realizad (Mecánica de la gamificación). \\\hline
EG10\labeltext{EG10}{feedback} & Retroalimentación inmediata & Utilización de una herramienta online para aportar retroalimentación inmediata a los alumnos al realizar los deberes \\\hline
EG11\labeltext{EG11}{fichas} & Economía de fichas & Tienda en la que poder comprar ciertos privilegios con los puntos obtenidos a lo largo de las sesiones. Descritos en \ref{tbl:tienda}. \\\hline
\end{tabular}
\end{table}
\FloatBarrier

\todo{Definir puntos de la tienda}

\begin{table}[hptb]
\centering
\caption{Artículos de la economía de fichas}
\label{tbl:tienda}
\begin{tabular}{|m{0.15\linewidth}|m{0.2\linewidth}|m{0.6\linewidth}|}
\hline
Puntos & Nombre & Descripción \\ \hline
$n$ & ExtraQuizz	& Aumentar en 1 el número de ejercicios que se pueden hacer en \textit{edmodo} para ganar puntos.\\\hline
$x^2·m$ & TeacherQuizz & $x$ preguntas directas (de sí o no) y precisas\footnotemark al docente en el examen\\\hline
%$t$ &  &  \\\hline
%$s$ &  &  \\\hline
%$r$ &  &  \\\hline
\end{tabular}
\footnotetext{Preguntas del tipo: ¿Este ejercicio está bien? es demasiado ambigua y no sería válida. Preguntas como: ¿Con el ejercicio así obtendría la máxima puntuación? sí sería válida.
%
El grado de precisión de una pregunta será competencia única del docente en el momento del examen.}
\end{table}
\FloatBarrier

\subsection{Evaluación}

\label{eval}
%
La evaluación que se propone para esta unidad didáctica es una evaluación tradicional, a través de una prueba objetiva incluida en \ref{examen} con la ponderación que estuviera establecida en la \gls{PDA} y el trabajo de clase que se medirá mediante los puntos y medallas obtenidos por cada alumno.
%
La obtención de la calificación numérica a partir de ellos se explica más adelante.


La evaluación a través de una prueba objetiva tiene varias ventajas:
%
la primera es la facilidad de la incorporación de esta unidad didáctica en un curso que ya funcione con metodologías tradicionales. 
%
No hace falta modificar los criterios de evaluación que se hubieran establecido para evaluar a los alumnos.
%
Además, permite y facilita la experimentación.
%
Se pueden comparar los resultados del aprendizaje mediante una prueba objetiva en un curso que ha participado en la gamificación frente a un curso que haya recibido las sesiones mediante otras metodologías.
% 
Así, el grupo de la gamificación sería el grupo experimental y el otro grupo sería el grupo de control. 
%
En el caso de que esta estrategia metodológica fuese adoptada por un único docente en un departamento, la posible experimentación permitiría iniciar un diálogo y debate sobre la incorporación de la gamificación por parte de otros docentes, incluso en otros cursos.

Dicho lo cual, sería injusto no comentar que la evaluación mediante una prueba objetiva no es la mejor manera de realizar una evaluación justa que asegure la adquisición de las competencias y el aprendizaje duradero los contenidos.
%
Se podrían utilizar rúbricas por ejemplo, que permitirían una evaluación mucho más exhaustiva y personalizada.
%
Sin embargo, se ha preferido optar por la prueba objetiva por todo lo comentado anteriormente.


Para establecer ponderación de las medallas y de los puntos de reputación para obtener la calificación del trabajo de clase se propone el siguiente sistema:
%
de los 10 puntos de actitud y trabajo en clase, hasta un máximo de 5 puntos se podrán obtener mediante logros, otorgando cada logro un punto\footnote{2 niveles de un mismo logro otorgarán 2 puntos} y hasta un máximo de 5 puntos se podrán obtener mediante puntos de reputación, mediante una regla de 3, correspondiendo 5 puntos con $\max$ puntos de reputación y 0 puntos con 0 puntos de reputación.




%%%%%%%%%%%%%%%%%%%%%%%%%%%%%%%%%%%%%%%%%%%%%%%%%%%%%%%%%%%%%%%%%%%%%%%%
%%%%%%%%%%%%%%%%%%%%%%%%%%%%%%%%%%%%%%%%%%%%%%%%%%%%%%%%%%%%%%%%%%%%%%%%
%%%%%%%%%%%%%%%%%%%%%%%%%%%%%%%%%%%%%%%%%%%%%%%%%%%%%%%%%%%%%%%%%%%%%%%%
%%%%%%%%%%%%%%%%%%%%%%%%%%%%%%%%%%%%%%%%%%%%%%%%%%%%%%%%%%%%%%%%%%%%%%%%
%%%%%%%%%%%%%%%%%%%%%%%%%%%%%%%%%%%%%%%%%%%%%%%%%%%%%%%%%%%%%%%%%%%%%%%%
%%%%%%%%%%%%%%%%%%%%%%%%%%%%%%%%%%%%%%%%%%%%%%%%%%%%%%%%%%%%%%%%%%%%%%%%
%%%%%%%%%%%%%%%%%%%%%%%%%%%%%%%%%%%%%%%%%%%%%%%%%%%%%%%%%%%%%%%%%%%%%%%%
%%%%%%%%%%%%%%%%%%%%%%%%%%%%%%%%%%%%%%%%%%%%%%%%%%%%%%%%%%%%%%%%%%%%%%%%
%%%%%%%%%%%%%%%%%%%%%%%%%%%%%%%%%%%%%%%%%%%%%%%%%%%%%%%%%%%%%%%%%%%%%%%%
%%%%%%%%%%%%%%%%%%%%%%%%%%%%%%%%%%%%%%%%%%%%%%%%%%%%%%%%%%%%%%%%%%%%%%%%
%%%%%%%%%%%%%%%%%%%%%%%%%%%%%%%%%%%%%%%%%%%%%%%%%%%%%%%%%%%%%%%%%%%%%%%%
%%%%%%%%%%%%%%%%%%%%%%%%%%%%%%%%%%%%%%%%%%%%%%%%%%%%%%%%%%%%%%%%%%%%%%%%
%%%%%%%%%%%%%%%%%%%%%%%%%%%%%%%%%%%%%%%%%%%%%%%%%%%%%%%%%%%%%%%%%%%%%%%%
%%%%%%%%%%%%%%%%%%%%%%%%%%%%%%%%%%%%%%%%%%%%%%%%%%%%%%%%%%%%%%%%%%%%%%%%
%%%%%%%%%%%%%%%%%%%%%%%%%%%%%%%%%%%%%%%%%%%%%%%%%%%%%%%%%%%%%%%%%%%%%%%%
%%%%%%%%%%%%%%%%%%%%%%%%%%%%%%%%%%%%%%%%%%%%%%%%%%%%%%%%%%%%%%%%%%%%%%%%
%%%%%%%%%%%%%%%%%%%%%%%%%%%%%%%%%%%%%%%%%%%%%%%%%%%%%%%%%%%%%%%%%%%%%%%%
%%%%%%%%%%%%%%%%%%%%%%%%%%%%%%%%%%%%%%%%%%%%%%%%%%%%%%%%%%%%%%%%%%%%%%%%
%%%%%%%%%%%%%%%%%%%%%%%%%%%%%%%%%%%%%%%%%%%%%%%%%%%%%%%%%%%%%%%%%%%%%%%%
%%%%%%%%%%%%%%%%%%%%%%%%%%%%%%%%%%%%%%%%%%%%%%%%%%%%%%%%%%%%%%%%%%%%%%%%
%%%%%%%%%%%%%%%%%%%%%%%%%%%%%%%%%%%%%%%%%%%%%%%%%%%%%%%%%%%%%%%%%%%%%%%%
%%%%%%%%%%%%%%%%%%%%%%%%%%%%%%%%%%%%%%%%%%%%%%%%%%%%%%%%%%%%%%%%%%%%%%%%
%%%%%%%%%%%%%%%%%%%%%%%%%%%%%%%%%%%%%%%%%%%%%%%%%%%%%%%%%%%%%%%%%%%%%%%%
%%%%%%%%%%%%%%%%%%%%%%%%%%%%%%%%%%%%%%%%%%%%%%%%%%%%%%%%%%%%%%%%%%%%%%%%
%%%%%%%%%%%%%%%%%%%%%%%%%%%%%%%%%%%%%%%%%%%%%%%%%%%%%%%%%%%%%%%%%%%%%%%%
%%%%%%%%%%%%%%%%%%%%%%%%%%%%%%%%%%%%%%%%%%%%%%%%%%%%%%%%%%%%%%%%%%%%%%%%
%%%%%%%%%%%%%%%%%%%%%%%%%%%%%%%%%%%%%%%%%%%%%%%%%%%%%%%%%%%%%%%%%%%%%%%%



\section{Evaluación de la gamificación}

En la sección \ref{eval} se ha esbozado una posible experimentación para evaluar la gamificación.
%
Para valorar si esta propuesta metodológica mejora otras propuestas metodológicas se puede llevar a cabo el siguiente experimento.

Se elige aleatoriamente la mitad de las clases de Matemáticas orientadas a las Ciencias Académicas de 3 de la ESO.
%
En esas clases se imparte utilizando la metodología específica del centro educativo.
%
En la otra mitad de las clases, se imparte la unidad didáctica utilizando la metodología propuesta.
%
De esta manera se puede comparar la gamificación como metodología con otras asignaturas utilizando como herramienta de medida la prueba objetiva.
%
Sería necesario, para asegurar la validez de la investigación, que los alumnos realizaran la prueba objetiva el mismo día y hubieran tenido el mismo número de sesiones de Matemáticas antes del examen.


Por otro lado, es fundamental saber si la gamificación tal como se ha llevado a cabo ha merecido la pena, qué aspectos son mejorables y cuáles hay que mantener.
%
Debido a las dificultades que se encuentran al querer plantear una experimentación utilizando esta unidad didáctica concreta se propone que el docente realice un informe a medida que van avanzando las sesiones 
y que al final se realice una autoevaluación buscando mejorar la propuesta.
%
Sería interesante contar con la retroalimentación por parte de los alumnos, ya que son los destinatarios de la gamificación.

