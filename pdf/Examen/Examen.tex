\documentclass[palatino,noprobframes]{CuartillaSafa}

\usepackage{fancyhdr}

\pagestyle{fancy}
\fancyhf{}

\renewcommand{\headrulewidth}{0.4pt}
\renewcommand{\headrule}{{
\vspace{0.3cm}\hrule width\headwidth height\headrulewidth \vskip-\headrulewidth}}

\renewcommand{\thesection}{\thechapter.\arabic{section}}

\chead{Examen de Polinomios}
%\cfoot{\thepage}




\setcounter{section}{1}

\title{Examen de Polinomios}
\date{24-02-2017}

% Paquetes adicionales

% --------------------

\renewcommand{\vec}[1]{\overrightarrow{#1}}

\begin{document}

\cabecera
\pagestyle{fancy}

\begin{problem}\textbf{(0.8 puntos)}
Señala todas las opciones correctas, siendo $\vec{u} = (u_1,u_2)$ el vector director de la recta $r: Ax+By+C = 0$ y siendo $m$ su pendiente:

a) $\displaystyle m=-\frac{u_1}{u_2}$\;\;\;\;\;\;\;\;\;
\textbf{b)} $\displaystyle m=\frac{u_2}{u_1}$\;\;\;\;\;\;\;\;\;
\textbf{c)} $\displaystyle m=-\frac{A}{B}$\;\;\;\;\;\;\;\;\;
d) $\displaystyle m=\frac{B}{A}$\;\;\;\;\;\;\;\;\;
\end{problem}


\begin{problem}\textbf{(1.2 puntos)} 
Une cada ecuación de la recta con su denominación:

\begin{tabular}{L{2.25cm}L{3cm}L{8cm}}
General 		&$\circ$&$\circ$\;\;$\displaystyle\frac{y-a_2}{u_2} = \frac{x-a_1}{u_1}$ \\\\
Explícita 		&$\circ$&$\circ$\;\;$\left\{\begin{array}{c} x=a_1+u_1·t\\y=a_2+u_2·t\end{array}\right.$ \\\\
Punto-pendiente &$\circ$&$\circ$\;\;$(x,y) = (a_1,a_2) + t·(u_1,u_2)$ \\\\
Continua 		&$\circ$&$\circ$\;\;$y-a_2 = m·(x-a_1)$ \\\\
Paramétrica 	&$\circ$&$\circ$\;\;$Ax+By+C=0$ \\\\
Vectorial 		&$\circ$&$\circ$\;\;$y=mx+n$ \\\\
\end{tabular}
\end{problem}
\vspace{-0.7cm}

\begin{problem}\textbf{ Posición relativa (3.5 puntos)}
\begin{enumerate}[a)]
	\vspace{-0.3cm}\item Determina la ecuación explícita de la recta $\vec{r}$ que pasa por $A(0,0)$ y $B(5,4)$ 
	\vspace{-0.3cm}\item Determina la recta $\vec{s}$ que pasa por $C(10,8)$ y tiene pendiente $m=0.8$ 
	\vspace{-0.3cm}\item Estudia la posición relativa de las rectas $\vec{r}$ y $\vec{s}$ de los apartados anteriores. 
	\vspace{-0.3cm}\item Justifica si son perpendiculares. 
\end{enumerate}
\textbf{Solución:}
\begin{enumerate}[a)]
	\item $\vec{u} = (5,4) \to m = \frac{4}{5} \to y=0.8·x+n \to 0=0.8·0+n \to n=0$\[y=0.8x\]
	\item Utilizamos la ecuación punto pendiente: $y-a_2 = m·(x-a_1) \to (y-8)=0.8·(x-10)$
	\[y=0.8x-8+8 \to y=0.8x\]
	\item Las rectas son coincidentes porque $m_r = m_s$ y $n_r = n_s$.
	\item No son perpendiculares porque $m_r·m_s = 0.8·0.8 \neq -1$.
\end{enumerate}


\end{problem}

\newpage

\begin{problem}\textbf{(2 puntos)}

Halla el ángulo que forman las rectas $\displaystyle \vec{r}: \frac{x-3}{2} = \frac{y+1}{-2}$ y $\vec{s}: \left\{\begin{array}{c} x=3\\y=1+2t\end{array}\right.$. 

\textbf{Solución:}
El ángulo que forman es el ángulo que forman sus vectores directores.

$u_r = (2,-2)$ y $u_s = (0,2)$

\[
\cos α = \frac{\vec{u_r}·\vec{u_s}}{|\vec{u_r}|·|\vec{u_s}|} = \frac{-4}{\sqrt{4+4}·\sqrt{4}} = \frac{-4}{4\sqrt{2}} = \frac{-\sqrt{2}}{2} \to α=\arccos{\frac{-\sqrt{2}}{2} = 135}
\]



\end{problem}

%\vspace{7cm}

\textbf{Elegir uno de los siguientes 2 ejercicios:}
\begin{problem}\textbf{(2.5 puntos)}

¿Están los puntos $A(2,2)$, $B(0,1)$ y $C(-2,0)$ alineados? Justifica tu respuesta.

\textbf{Solución 1:}

$\vec{AB} = (-2,-1)$ y $\vec{BC} = (-2,-1)$. Son linealmente dependientes y tienen un punto en común, por lo que las rectas que forman son coincidentes, por lo que los 3 puntos están alineados.


\textbf{Solución 2:}
Recta $\vec{r}$ que pasa por A y B: \[
\frac{y-2}{2-1} = \frac{x-2}{2-0} \to \frac{y-2}{1} = \frac{x-2}{2}
\]
Comprobamos si $C\in \vec{r}$: 
\[
	\frac{-2-2}{2} = \frac{0-2}{1} \to \frac{-4}{2} = \frac{-2}{1} \to C\in\vec{r}
\]
Los 3 puntos están alineados.

\end{problem}
\begin{problem}\textbf{(2.5 puntos)}

Determina la recta perpendicular a $\vec{r}: y=3x$ en el punto $P(3,1)$. ¿Cuál es el punto de corte de $\vec{r}$ y su perpendicular?

Queremos $\vec{s}\perp\vec{r}$, por lo que $m_s = \frac{-1}{m_r} = \frac{-1}{3}$.

La ecuación punto pendiente de la recta es:

\[
	y-1 = \frac{-1}{3}(x-3)
\]

Para resolver el punto de corte comprobamos que $P(3,1)$ pertenece a $\vec{r}$ y a $\vec{s}$, por lo que tiene que ser el punto de corte.

\[
P\in\vec{r} \to 3 = 3·1 \to 3=3 \text{ verifica la ecuación.}
\]

\[
P\in\vec{s} \to 1-1 = \frac{-1}{3}·(3-3) \to 0=0 \text{ verifica la ecuación.}
\]

Como las 2 rectas son secantes sólo tienen un punto en común. Como $P$ pertenece a las 2 rectas, $P$ es el punto de corte.

\end{problem}






%% Apendices (ejercicios, examenes)


\end{document}
