\documentclass[palatino,noprobframes]{CuartillaSafa}

\usepackage{fancyhdr}

\pagestyle{fancy}
\fancyhf{}

\renewcommand{\headrulewidth}{0.4pt}
\renewcommand{\headrule}{{
\vspace{0.3cm}\hrule width\headwidth height\headrulewidth \vskip-\headrulewidth}}

\renewcommand{\thesection}{\thechapter.\arabic{section}}

\chead{Examen de Polinomios}
%\cfoot{\thepage}




\setcounter{section}{1}

\title{Examen de Polinomios}
\date{24-02-2017}

% Paquetes adicionales

% --------------------

\renewcommand{\vec}[1]{\overrightarrow{#1}}

\begin{document}

\cabecera
\pagestyle{fancy}

\begin{problem}\textbf{1 punto}
Simplifica y ordena los siguientes polinomios reduciendo los términos que sean semejantes: (0,5 puntos por apartado)

a) $4x^2-x^3-4x+7+x^2-4*2x*6x^3-5x^2 = 5x^3-2x+3$

b) $\frac{1}{2}x^3+3x-\frac{5}{4}-2x^3+2-4x = \frac{-3}{2}x^3-x+\frac{3}{4}$

\end{problem}

\begin{problem}\textbf{3 puntos}

Dados los siguientes polinomios

\[
P(x) = 4x^5+x^3-2x^2+5x - 7 
Q(x) = -x^3+4x^2-2x-1 
R(x) = 2x^2-x+3
\]

efectúa las siguientes operaciones:

a) $P(x) +Q(x)$ (0,5 puntos) $= 4x^5+x^2+3x-8$

b) $R(x) - P(x)$ (0.5 puntos) $=-4x^5-x^3+4x^2-6x+10$

c) $Q(x)\cdot R(x)$ (1 punto) $=2x^5+7x^4-10x^3+9x^2-5x-3$

d) $\frac{P(x)}{R(x)}$ (1 punto) $= 2x^3+x^2-2x-\frac{7}{2}$

\end{problem}

\begin{problem}\textbf{(2 puntos ; 0,5 cada apartado)}

Saca factor común en las siguientes expresiones:

a) $4x^3+8x^4-6x^2 = 2x^2(2x+4x^2-3)$

b) $15x^2z-6xz^2-3xz+9x^2z^2 = 3xz(5x-2z-1+3xz)$

c) $18z^7y^2-9x^5y^3+27x^3y^4 = 9x^3y^2(2x^4-x^2y+3y^2)$

d) $2abc-2bc-2bcd = 2bc(a-1-d)$

\end{problem}


\begin{problem}\textbf{(2 puntos; 0,5 cada apartado)}

Desarrolla las siguientes expresiones utilizando las identidades notables:

a) $(2x+3)^2 = 4x^2+9+12x$

b) $(4-5y)^2 = 16-40y+25y^2$

c) $(5x-4y)(5x+4y) = 25x^2-16y^2$

d) $\left(3x^2-2x^3\right)^2 = 9x^4 - 12x^5 + 4x^6$
	
\end{problem}


\begin{problem}\textbf{(1 punto)}

	Realiza la siguiente división utilizando el algoritmo de Ruffini, indicando el cociente y el resto.

	$(3x^6+2x^5+x^4-x^2-3) : (x+1)$

	Cociente: $3x^5-x^4+2x^3-2x^2+x-1$ y Resto: $-2$

\end{problem}

\begin{problem}\textbf{(1 punto)}

	Factoriza el siguente polinomio:

	$x^4-3x^2+2x = (x-1)^2(x+2)x$


\end{problem}

%% Apendices (ejercicios, examenes)


\end{document}
